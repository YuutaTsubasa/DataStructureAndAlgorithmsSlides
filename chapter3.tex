\documentclass{beamer}
\usetheme{Madrid}

\usepackage{xeCJK}
\usepackage{fontspec}
\setmainfont[
  Path = fonts/,
  UprightFont = *-Regular,
  BoldFont = *-Bold,
  ItalicFont = *-Medium,
  BoldItalicFont = *-Bold
]{NotoSansTC}
\setCJKmainfont[
  Path = fonts/,
  UprightFont = *-Regular,
  BoldFont = *-Bold,
  ItalicFont = *-Medium,
  BoldItalicFont = *-Bold
]{NotoSansTC}

\usepackage{amsmath}
\usepackage{graphicx}
\usepackage{listings}
\usepackage{xcolor}

\lstdefinestyle{cppstyle}{
  language=C++,
  basicstyle=\ttfamily\small,
  keywordstyle=\color{blue}\bfseries,
  commentstyle=\color{gray},
  stringstyle=\color{orange},
  numbers=left,
  numberstyle=\tiny\color{gray},
  stepnumber=1,
  numbersep=5pt,
  frame=single,
  breaklines=true,
  showstringspaces=false,
  tabsize=2
}

\begin{document}

\title{資料結構與演算法入門:第 3 章}
\subtitle{分而治之 (Divide And Conquer)}
\author{悠太翼 Yuuta Tsubasa}
\date{\today}

\frame{\titlepage}

\begin{frame}{什麼是分而治之?}
\begin{itemize}
    \item 分而治之(Divide and Conquer)是一種重要的演算法設計策略
    \item 核心思想:將大問題拆解成多個較小的子問題
    \item 遞迴地解決子問題,再將結果合併
\end{itemize}

\vspace{1em}
\begin{block}{分而治之的三個步驟}
\begin{enumerate}
    \item \textbf{分解(Divide):}將原問題分解成若干個規模較小的子問題
    \item \textbf{解決(Conquer):}遞迴地解決各個子問題
    \item \textbf{合併(Combine):}將子問題的解合併成原問題的解
\end{enumerate}
\end{block}
\end{frame}

\begin{frame}{分而治之的時間複雜度分析}
\begin{itemize}
    \item 分而治之演算法的時間複雜度通常可以用遞迴關係式表示
    \item 一般形式:$T(n) = aT(n/b) + f(n)$
    \begin{itemize}
        \item $a$:子問題的數量
        \item $n/b$:每個子問題的規模
        \item $f(n)$:分解和合併的時間
    \end{itemize}
    \item 可使用主定理(Master Theorem)求解
\end{itemize}
\end{frame}

\section{Map \& Reduce}

\begin{frame}{Map \& Reduce 概念}
\begin{itemize}
    \item Map 和 Reduce 是函數式程式設計中的重要概念
    \item 也是分而治之思想的體現
\end{itemize}

\vspace{1em}
\begin{block}{Map 操作}
\begin{itemize}
    \item 對每個資料元素套用相同的函數
    \item 產生新的資料集合
    \item 可以平行化處理
\end{itemize}
\end{block}

\vspace{0.5em}
\begin{block}{Reduce 操作}
\begin{itemize}
    \item 將資料集合歸約成單一結果
    \item 使用結合律的操作(如加法、乘法)
    \item 可以遞迴地進行
\end{itemize}
\end{block}
\end{frame}

\begin{frame}{Map \& Reduce 範例:陣列求和}
\begin{itemize}
    \item 使用 Map \& Reduce 計算陣列總和
    \item Map:每個元素乘以 1(恆等操作)
    \item Reduce:將所有元素相加
\end{itemize}

\vspace{1em}
\textbf{待補充:程式碼範例}
\end{frame}

\section{分而治之範例}

\begin{frame}{範例一:合併排序 (Merge Sort)}
\begin{itemize}
    \item 經典的分而治之演算法
    \item 分解:將陣列分成兩半
    \item 解決:遞迴地排序兩個子陣列
    \item 合併:將兩個已排序的子陣列合併
\end{itemize}

\vspace{1em}
\textbf{時間複雜度:}$O(n \log n)$\\
\textbf{空間複雜度:}$O(n)$

\vspace{1em}
\textbf{待補充:程式碼實作}
\end{frame}

\begin{frame}{範例二:快速排序 (Quick Sort)}
\begin{itemize}
    \item 另一個重要的分而治之排序演算法
    \item 選擇基準點(pivot)
    \item 分割:將元素分為小於和大於基準點的兩組
    \item 遞迴地排序兩個子陣列
\end{itemize}

\vspace{1em}
\textbf{平均時間複雜度:}$O(n \log n)$\\
\textbf{最壞時間複雜度:}$O(n^2)$\\
\textbf{空間複雜度:}$O(\log n)$

\vspace{1em}
\textbf{待補充:程式碼實作}
\end{frame}

\begin{frame}{範例三:二分搜尋 (Binary Search)}
\begin{itemize}
    \item 在已排序陣列中搜尋特定元素
    \item 每次比較中間元素,排除一半的搜尋空間
    \item 遞迴地在剩餘區間中搜尋
\end{itemize}

\vspace{1em}
\textbf{時間複雜度:}$O(\log n)$\\
\textbf{空間複雜度:}$O(1)$(迭代版本)或 $O(\log n)$(遞迴版本)

\vspace{1em}
\textbf{待補充:程式碼實作}
\end{frame}

\begin{frame}{範例四:最大子陣列問題}
\begin{itemize}
    \item 找出陣列中連續子陣列的最大和
    \item 分解:將陣列分成左、右兩半
    \item 考慮三種情況:
    \begin{itemize}
        \item 最大子陣列完全在左半部
        \item 最大子陣列完全在右半部
        \item 最大子陣列跨越中間點
    \end{itemize}
\end{itemize}

\vspace{1em}
\textbf{時間複雜度:}$O(n \log n)$

\vspace{1em}
\textbf{待補充:程式碼實作與圖解}
\end{frame}

\begin{frame}{分而治之的優缺點}
\begin{center}
\renewcommand{\arraystretch}{1.4}
\begin{tabular}{|>{\centering\arraybackslash}m{3cm}|>{\raggedright\arraybackslash}m{6.5cm}|}
\hline
\textbf{優點} & 
易於理解和實作 \newline
能夠有效利用遞迴結構 \newline
許多問題都有高效的分而治之解法 \newline
容易平行化 \\
\hline
\textbf{缺點} & 
可能有額外的空間開銷 \newline
遞迴深度可能很大 \newline
子問題間可能有重複計算 \\
\hline
\end{tabular}
\end{center}
\end{frame}

\begin{frame}{本章總結}
\begin{itemize}
    \item 分而治之是重要的演算法設計技巧
    \item 核心思想:分解、解決、合併
    \item Map \& Reduce 是分而治之的函數式表達
    \item 經典應用:
    \begin{itemize}
        \item 排序演算法(合併排序、快速排序)
        \item 搜尋演算法(二分搜尋)
        \item 數值計算問題
    \end{itemize}
    \item 時間複雜度分析可使用主定理
\end{itemize}

\vspace{1em}
\begin{center}
    \textbf{掌握分而治之,解決問題事半功倍!}
\end{center}
\end{frame}

\end{document}