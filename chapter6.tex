\documentclass{beamer}
\usetheme{Madrid}

\usepackage{xeCJK}
\usepackage{fontspec}
\setmainfont[
  Path = fonts/,
  UprightFont = *-Regular,
  BoldFont = *-Bold,
  ItalicFont = *-Medium,
  BoldItalicFont = *-Bold
]{NotoSansTC}
\setCJKmainfont[
  Path = fonts/,
  UprightFont = *-Regular,
  BoldFont = *-Bold,
  ItalicFont = *-Medium,
  BoldItalicFont = *-Bold
]{NotoSansTC}

\usepackage{amsmath}
\usepackage{graphicx}
\usepackage{listings}
\usepackage{xcolor}
\usepackage{tikz}

\lstdefinestyle{cppstyle}{
  language=C++,
  basicstyle=\ttfamily\small,
  keywordstyle=\color{blue}\bfseries,
  commentstyle=\color{gray},
  stringstyle=\color{orange},
  numbers=left,
  numberstyle=\tiny\color{gray},
  stepnumber=1,
  numbersep=5pt,
  frame=single,
  breaklines=true,
  showstringspaces=false,
  tabsize=2
}

\begin{document}

\title{資料結構與演算法入門:第 6 章}
\subtitle{堆疊與佇列 (Stack \& Queue)}
\author{悠太翼 Yuuta Tsubasa}
\date{\today}

\frame{\titlepage}

\begin{frame}{什麼是堆疊?}
\begin{itemize}
    \item 堆疊(Stack)是一種 LIFO(Last In, First Out)資料結構
    \item 只能在一端進行插入和刪除操作,這一端稱為「頂端」
    \item 就像疊盤子:最後放上去的盤子最先被拿走
\end{itemize}

\vspace{1em}
\begin{block}{基本操作}
\begin{itemize}
    \item \textbf{Push:}將元素推入堆疊頂端
    \item \textbf{Pop:}從堆疊頂端移除元素
    \item \textbf{Top/Peek:}查看堆疊頂端元素(不移除)
    \item \textbf{Empty:}檢查堆疊是否為空
\end{itemize}
\end{block}

\vspace{1em}
\begin{center}
\textbf{所有操作的時間複雜度都是 $O(1)$}
\end{center}
\end{frame}

\begin{frame}{堆疊的實作方式}
\begin{itemize}
    \item \textbf{陣列實作:}
    \begin{itemize}
        \item 使用陣列和一個指標記錄頂端位置
        \item 簡單高效,但大小固定
    \end{itemize}
    \item \textbf{連結串列實作:}
    \begin{itemize}
        \item 每個節點指向下一個節點
        \item 動態大小,但需要額外指標空間
    \end{itemize}
\end{itemize}

\vspace{1em}
\textbf{待補充:兩種實作方式的程式碼}
\end{frame}

\begin{frame}{什麼是佇列?}
\begin{itemize}
    \item 佇列(Queue)是一種 FIFO(First In, First Out)資料結構
    \item 在一端插入(後端),在另一端刪除(前端)
    \item 就像排隊:先到的人先服務
\end{itemize}

\vspace{1em}
\begin{block}{基本操作}
\begin{itemize}
    \item \textbf{Enqueue:}將元素加入佇列後端
    \item \textbf{Dequeue:}從佇列前端移除元素
    \item \textbf{Front:}查看佇列前端元素(不移除)
    \item \textbf{Empty:}檢查佇列是否為空
\end{itemize}
\end{block}

\vspace{1em}
\begin{center}
\textbf{所有操作的時間複雜度都是 $O(1)$}
\end{center}
\end{frame}

\begin{frame}{佇列的實作方式}
\begin{itemize}
    \item \textbf{陣列實作:}
    \begin{itemize}
        \item 使用兩個指標記錄前端和後端位置
        \item 需要處理「假滿」問題 → 環狀陣列
    \end{itemize}
    \item \textbf{連結串列實作:}
    \begin{itemize}
        \item 維護前端和後端指標
        \item 動態大小,實作相對簡單
    \end{itemize}
\end{itemize}

\vspace{1em}
\textbf{環狀陣列:}當指標到達陣列末端時,回到開頭繼續使用

\vspace{1em}
\textbf{待補充:環狀陣列圖解與實作程式碼}
\end{frame}

\section{雙端佇列 (Deque)}

\begin{frame}{雙端佇列 (Double-ended Queue, Deque)}
\begin{itemize}
    \item 可以在兩端進行插入和刪除操作的資料結構
    \item 結合了堆疊和佇列的特性
    \item 更加靈活的資料結構
\end{itemize}

\vspace{1em}
\begin{block}{基本操作}
\begin{itemize}
    \item \textbf{Push Front:}在前端插入元素
    \item \textbf{Push Back:}在後端插入元素
    \item \textbf{Pop Front:}從前端刪除元素
    \item \textbf{Pop Back:}從後端刪除元素
\end{itemize}
\end{block}

\vspace{1em}
\textbf{所有操作的時間複雜度都是 $O(1)$}
\end{frame}

\begin{frame}{Deque 的實作方式}
\begin{itemize}
    \item \textbf{雙向連結串列:}
    \begin{itemize}
        \item 維護頭尾指標
        \item 操作簡單直觀
    \end{itemize}
    \item \textbf{環狀陣列:}
    \begin{itemize}
        \item 使用動態陣列,支援兩端操作
        \item 較複雜但記憶體效率更高
    \end{itemize}
    \item \textbf{分塊實作:}
    \begin{itemize}
        \item 將資料分成多個區塊
        \item STL deque 的實作方式
    \end{itemize}
\end{itemize}

\vspace{1em}
\textbf{待補充:各種實作方式的詳細說明}
\end{frame}

\section{堆疊與佇列的應用}

\begin{frame}{堆疊應用一:括號匹配}
\begin{itemize}
    \item 檢查括號是否正確配對:()、[]、\{\}
    \item 演算法:
    \begin{enumerate}
        \item 遇到開括號就推入堆疊
        \item 遇到閉括號就檢查堆疊頂端是否匹配
        \item 最後堆疊應該為空
    \end{enumerate}
\end{itemize}

\vspace{1em}
\textbf{範例:}
\begin{itemize}
    \item "()[]\{\}" → 正確
    \item "([)]" → 錯誤
    \item "(((" → 錯誤
\end{itemize}

\vspace{1em}
\textbf{待補充:括號匹配程式碼實作}
\end{frame}

\begin{frame}{堆疊應用二:中綴轉後綴表達式}
\begin{itemize}
    \item \textbf{中綴表達式:}$(A + B) \times (C - D)$
    \item \textbf{後綴表達式:}$A B + C D - \times$
    \item 使用 Shunting Yard 演算法
\end{itemize}

\vspace{1em}
\begin{block}{轉換步驟}
\begin{enumerate}
    \item 遇到運算元,直接輸出
    \item 遇到運算子,考慮優先順序推入堆疊或輸出
    \item 遇到左括號,推入堆疊
    \item 遇到右括號,輸出至左括號為止
\end{enumerate}
\end{block}

\vspace{1em}
\textbf{待補充:詳細演算法步驟與程式碼}
\end{frame}

\begin{frame}{堆疊應用三:函數呼叫與遞迴}
\begin{itemize}
    \item 程式語言使用堆疊管理函數呼叫
    \item 每次函數呼叫會推入新的「活化記錄」
    \item 包含參數、區域變數、返回位址
    \item 函數結束時彈出堆疊
\end{itemize}

\vspace{1em}
\begin{block}{堆疊溢位 (Stack Overflow)}
\begin{itemize}
    \item 遞迴深度過大
    \item 無限遞迴
    \item 局部變數佔用過多空間
\end{itemize}
\end{block}

\vspace{1em}
\textbf{待補充:函數呼叫堆疊圖解}
\end{frame}

\begin{frame}{佇列應用一:廣度優先搜尋 (BFS)}
\begin{itemize}
    \item BFS 使用佇列來實作
    \item 確保按照距離遞增的順序訪問節點
    \item 適用於最短路徑問題
\end{itemize}

\vspace{1em}
\begin{block}{BFS 演算法步驟}
\begin{enumerate}
    \item 將起始節點加入佇列
    \item 當佇列不為空時:
    \begin{itemize}
        \item 從佇列前端取出節點
        \item 處理該節點
        \item 將未訪問的鄰居加入佇列
    \end{itemize}
\end{enumerate}
\end{block}

\vspace{1em}
\textbf{待補充:BFS 程式碼實作與圖解}
\end{frame}

\begin{frame}{佇列應用二:作業調度}
\begin{itemize}
    \item 作業系統的程序調度
    \item 印表機工作佇列
    \item 網路封包緩衝
\end{itemize}

\vspace{1em}
\begin{block}{Round Robin 調度}
\begin{itemize}
    \item 每個程序分配固定時間片
    \item 時間用完就移到佇列末端
    \item 確保公平性
\end{itemize}
\end{block}

\vspace{1em}
\textbf{待補充:調度演算法模擬}
\end{frame}

\begin{frame}{Deque 應用:滑動視窗最值}
\begin{itemize}
    \item 問題:給定陣列和視窗大小 $k$,找出每個視窗的最大值
    \item 使用 deque 維護可能的最大值候選
    \item 保持 deque 中的元素遞減排序
\end{itemize}

\vspace{1em}
\begin{block}{演算法步驟}
\begin{enumerate}
    \item 移除超出視窗範圍的元素
    \item 移除比當前元素小的尾端元素
    \item 將當前元素加入 deque 尾端
    \item deque 前端就是視窗最大值
\end{enumerate}
\end{block}

\vspace{1em}
\textbf{時間複雜度:}$O(n)$

\vspace{1em}
\textbf{待補充:滑動視窗演算法圖解與程式碼}
\end{frame}

\begin{frame}{Stack vs Queue vs Deque 比較}
\begin{center}
\renewcommand{\arraystretch}{1.4}
\begin{tabular}{|>{\centering\arraybackslash}m{2.5cm}|>{\centering\arraybackslash}m{2cm}|>{\centering\arraybackslash}m{2cm}|>{\centering\arraybackslash}m{2.5cm}|}
\hline
\textbf{資料結構} & \textbf{插入/刪除} & \textbf{存取模式} & \textbf{主要應用} \\
\hline
Stack & 單端 & LIFO & 函數呼叫、表達式計算 \\
\hline
Queue & 雙端 & FIFO & BFS、調度 \\
\hline
Deque & 雙端 & 靈活 & 滑動視窗、雙向搜尋 \\
\hline
\end{tabular}
\end{center}

\vspace{1em}
\begin{itemize}
    \item 選擇依據問題的存取模式
    \item 考慮實作的複雜度和效率
    \item Stack 最簡單,Deque 最靈活
\end{itemize}
\end{frame}

\begin{frame}{本章總結}
\begin{itemize}
    \item \textbf{Stack(堆疊):}LIFO,適合需要回溯的問題
    \item \textbf{Queue(佇列):}FIFO,適合按順序處理的問題
    \item \textbf{Deque(雙端佇列):}兩端操作,更加靈活
    \item \textbf{重要應用:}
    \begin{itemize}
        \item Stack:括號匹配、表達式轉換、函數呼叫
        \item Queue:BFS、作業調度
        \item Deque:滑動視窗問題
    \end{itemize}
    \item 實作方式影響效率,需要根據需求選擇
\end{itemize}

\vspace{1em}
\begin{center}
    \textbf{線性資料結構的基石,掌握它們解決更多問題!}
\end{center}
\end{frame}

\end{document}