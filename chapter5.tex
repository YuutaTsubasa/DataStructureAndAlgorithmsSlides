\documentclass{beamer}
\usetheme{Madrid}

\usepackage{xeCJK}
\usepackage{fontspec}
\setmainfont[
  Path = fonts/,
  UprightFont = *-Regular,
  BoldFont = *-Bold,
  ItalicFont = *-Medium,
  BoldItalicFont = *-Bold
]{NotoSansTC}
\setCJKmainfont[
  Path = fonts/,
  UprightFont = *-Regular,
  BoldFont = *-Bold,
  ItalicFont = *-Medium,
  BoldItalicFont = *-Bold
]{NotoSansTC}

\usepackage{amsmath}
\usepackage{graphicx}
\usepackage{listings}
\usepackage{xcolor}
\usepackage{tikz}

\lstdefinestyle{cppstyle}{
  language=C++,
  basicstyle=\ttfamily\small,
  keywordstyle=\color{blue}\bfseries,
  commentstyle=\color{gray},
  stringstyle=\color{orange},
  numbers=left,
  numberstyle=\tiny\color{gray},
  stepnumber=1,
  numbersep=5pt,
  frame=single,
  breaklines=true,
  showstringspaces=false,
  tabsize=2
}

\begin{document}

\title{資料結構與演算法入門:第 5 章}
\subtitle{進階陣列與連結串列 (Advanced Array \& Linked List)}
\author{悠太翼 Yuuta Tsubasa}
\date{\today}

\frame{\titlepage}

\begin{frame}{回顧:基本陣列}
\begin{itemize}
    \item 陣列是最基礎的資料結構
    \item 特點:
    \begin{itemize}
        \item 元素在記憶體中連續存放
        \item 支援隨機存取($O(1)$ 時間)
        \item 大小通常固定
    \end{itemize}
    \item 缺點:
    \begin{itemize}
        \item 插入和刪除操作代價高($O(n)$)
        \item 空間利用率可能不佳
    \end{itemize}
\end{itemize}

\vspace{1em}
\begin{center}
\textbf{如何改善這些限制?}
\end{center}
\end{frame}

\section{自動增長陣列}

\begin{frame}{自動增長陣列 (Dynamic Array)}
\begin{itemize}
    \item 也稱為動態陣列、可變陣列
    \item 可以在執行時期動態調整大小
    \item 常見實作:C++ 的 vector、Java 的 ArrayList、Python 的 list
\end{itemize}

\vspace{1em}
\begin{block}{核心概念}
\begin{itemize}
    \item \textbf{容量 (Capacity):}實際分配的記憶體大小
    \item \textbf{大小 (Size):}目前存放的元素數量
    \item 當容量不足時,重新分配更大的記憶體空間
\end{itemize}
\end{block}
\end{frame}

\begin{frame}{動態陣列的擴充策略}
\begin{itemize}
    \item \textbf{倍增策略:}容量不足時擴充為原來的 2 倍
    \item \textbf{優點:}分攤時間複雜度低
    \item \textbf{過程:}
    \begin{enumerate}
        \item 分配新的記憶體空間(通常是原來的 2 倍)
        \item 將舊資料複製到新空間
        \item 釋放舊的記憶體空間
        \item 更新指標
    \end{enumerate}
\end{itemize}

\vspace{1em}
\textbf{分攤分析:}雖然單次擴充需要 $O(n)$ 時間,但平均每次插入只需 $O(1)$ 時間

\vspace{1em}
\textbf{待補充:動態陣列實作與分攤分析詳解}
\end{frame}

\section{單向連結串列}

\begin{frame}{單向連結串列 (Single Linked List)}
\begin{itemize}
    \item 由節點(Node)組成的線性資料結構
    \item 每個節點包含:
    \begin{itemize}
        \item 資料 (Data)
        \item 指向下一個節點的指標 (Next)
    \end{itemize}
    \item 第一個節點稱為頭節點 (Head)
    \item 最後一個節點的 Next 指標為 NULL
\end{itemize}

\vspace{1em}
\begin{center}
\textbf{待補充:單向連結串列圖解}
\end{center}
\end{frame}

\begin{frame}{單向連結串列的基本操作}
\begin{itemize}
    \item \textbf{插入 (Insert):}
    \begin{itemize}
        \item 頭部插入:$O(1)$
        \item 尾部插入:$O(n)$(需要遍歷到最後)
        \item 指定位置插入:$O(n)$
    \end{itemize}
    \item \textbf{刪除 (Delete):}
    \begin{itemize}
        \item 頭部刪除:$O(1)$
        \item 指定節點刪除:$O(n)$
    \end{itemize}
    \item \textbf{搜尋 (Search):}$O(n)$
    \item \textbf{存取 (Access):}$O(n)$
\end{itemize}

\vspace{1em}
\textbf{待補充:各種操作的程式碼實作}
\end{frame}

\section{雙向連結串列}

\begin{frame}{雙向連結串列 (Double Linked List)}
\begin{itemize}
    \item 每個節點包含三個部分:
    \begin{itemize}
        \item 資料 (Data)
        \item 指向前一個節點的指標 (Previous)
        \item 指向下一個節點的指標 (Next)
    \end{itemize}
    \item 可以雙向遍歷
    \item 刪除操作更加高效
\end{itemize}

\vspace{1em}
\begin{block}{優點}
\begin{itemize}
    \item 可以從任意節點向前或向後遍歷
    \item 刪除指定節點時不需要額外的前驅節點資訊
\end{itemize}
\end{block}

\vspace{1em}
\begin{block}{缺點}
\begin{itemize}
    \item 每個節點需要額外的指標空間
    \item 插入和刪除操作需要更新更多指標
\end{itemize}
\end{block}
\end{frame}

\begin{frame}{雙向連結串列的操作}
\begin{itemize}
    \item \textbf{插入操作:}需要更新 4 個指標
    \begin{itemize}
        \item 新節點的 prev 和 next
        \item 前驅節點的 next
        \item 後繼節點的 prev
    \end{itemize}
    \item \textbf{刪除操作:}需要更新 2 個指標
    \begin{itemize}
        \item 前驅節點的 next 指向後繼節點
        \item 後繼節點的 prev 指向前驅節點
    \end{itemize}
\end{itemize}

\vspace{1em}
\textbf{待補充:雙向連結串列圖解與程式碼實作}
\end{frame}

\section{環狀連結串列}

\begin{frame}{環狀連結串列 (Cyclic Linked List)}
\begin{itemize}
    \item 最後一個節點的 next 指標指向第一個節點
    \item 形成一個環狀結構
    \item 可以是單向環狀或雙向環狀
\end{itemize}

\vspace{1em}
\begin{block}{特點}
\begin{itemize}
    \item 沒有明確的開始和結束
    \item 可以從任意節點遍歷整個串列
    \item 需要特別注意避免無窮迴圈
\end{itemize}
\end{block}

\vspace{1em}
\begin{block}{應用場景}
\begin{itemize}
    \item 循環任務調度
    \item 音樂播放器的循環播放
    \item 遊戲中的輪流機制
\end{itemize}
\end{block}
\end{frame}

\begin{frame}{環狀連結串列的實作要點}
\begin{itemize}
    \item \textbf{建立環狀結構:}
    \begin{itemize}
        \item 最後一個節點的 next 指向頭節點
        \item 雙向環狀中,頭節點的 prev 指向最後一個節點
    \end{itemize}
    \item \textbf{遍歷:}需要記錄起始節點,避免無窮迴圈
    \item \textbf{插入和刪除:}需要考慮邊界情況
\end{itemize}

\vspace{1em}
\textbf{環的檢測:}Floyd's Cycle Detection Algorithm (龜兔賽跑演算法)

\vspace{1em}
\textbf{待補充:環狀連結串列實作與環檢測演算法}
\end{frame}

\section{連結串列應用舉例}

\begin{frame}{舉例一:LRU Cache 實作}
\begin{itemize}
    \item LRU (Least Recently Used) 快取機制
    \item 使用雙向連結串列 + 雜湊表
    \item 雙向連結串列維護使用順序
    \item 雜湊表提供 $O(1)$ 查找
\end{itemize}

\vspace{1em}
\begin{block}{操作流程}
\begin{itemize}
    \item \textbf{Get:}移動節點到頭部
    \item \textbf{Put:}插入新節點到頭部,必要時刪除尾部節點
\end{itemize}
\end{block}

\vspace{1em}
\textbf{待補充:LRU Cache 完整實作}
\end{frame}

\begin{frame}{舉例二:多項式運算}
\begin{itemize}
    \item 使用連結串列表示稀疏多項式
    \item 每個節點存放係數和指數
    \item 只儲存非零項,節省空間
\end{itemize}

\vspace{1em}
\textbf{範例:}$3x^5 + 2x^3 - x + 7$

\vspace{1em}
\begin{center}
\textbf{待補充:多項式連結串列表示與加法運算}
\end{center}
\end{frame}

\begin{frame}{舉例三:約瑟夫問題 (Josephus Problem)}
\begin{itemize}
    \item \textbf{問題:}$n$ 個人圍成圓圈,從第一個人開始數,每數到第 $k$ 個人就將其移除,求最後剩下的人的位置
    \item \textbf{解法:}使用環狀連結串列模擬過程
    \item 每次移除一個節點,直到只剩一個節點
\end{itemize}

\vspace{1em}
\textbf{時間複雜度:}$O(nk)$\\
\textbf{空間複雜度:}$O(n)$

\vspace{1em}
\textbf{數學解法:}$J(n, k) = (J(n-1, k) + k) \bmod n$

\vspace{1em}
\textbf{待補充:約瑟夫問題的連結串列解法與數學公式推導}
\end{frame}

\begin{frame}{陣列 vs 連結串列比較}
\begin{center}
\renewcommand{\arraystretch}{1.4}
\begin{tabular}{|>{\centering\arraybackslash}m{2.5cm}|>{\centering\arraybackslash}m{2cm}|>{\centering\arraybackslash}m{2cm}|>{\centering\arraybackslash}m{2.5cm}|}
\hline
\textbf{操作} & \textbf{陣列} & \textbf{連結串列} & \textbf{動態陣列} \\
\hline
隨機存取 & $O(1)$ & $O(n)$ & $O(1)$ \\
\hline
頭部插入 & $O(n)$ & $O(1)$ & $O(n)$ \\
\hline
尾部插入 & $O(1)$ & $O(n)$ & $O(1)$ 分攤 \\
\hline
頭部刪除 & $O(n)$ & $O(1)$ & $O(n)$ \\
\hline
記憶體使用 & 連續 & 分散 & 連續 \\
\hline
額外空間 & 無 & 指標 & 可能未滿 \\
\hline
\end{tabular}
\end{center}
\end{frame}

\begin{frame}{本章總結}
\begin{itemize}
    \item \textbf{自動增長陣列:}結合陣列和動態調整的優點
    \item \textbf{連結串列類型:}
    \begin{itemize}
        \item 單向連結串列:簡單,插入刪除靈活
        \item 雙向連結串列:可雙向遍歷,刪除更高效
        \item 環狀連結串列:適合循環性質的問題
    \end{itemize}
    \item \textbf{選擇準則:}
    \begin{itemize}
        \item 需要隨機存取 → 陣列或動態陣列
        \item 頻繁插入刪除 → 連結串列
        \item 記憶體使用敏感 → 考慮額外指標開銷
    \end{itemize}
\end{itemize}

\vspace{1em}
\begin{center}
    \textbf{理解不同資料結構的特性,選擇最適合的工具!}
\end{center}
\end{frame}

\end{document}