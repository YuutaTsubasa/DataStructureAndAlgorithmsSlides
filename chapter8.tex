\documentclass{beamer}
\usetheme{Madrid}

\usepackage{xeCJK}
\usepackage{fontspec}
\setmainfont[
  Path = fonts/,
  UprightFont = *-Regular,
  BoldFont = *-Bold,
  ItalicFont = *-Medium,
  BoldItalicFont = *-Bold
]{NotoSansTC}
\setCJKmainfont[
  Path = fonts/,
  UprightFont = *-Regular,
  BoldFont = *-Bold,
  ItalicFont = *-Medium,
  BoldItalicFont = *-Bold
]{NotoSansTC}

\usepackage{amsmath}
\usepackage{graphicx}
\usepackage{listings}
\usepackage{xcolor}
\usepackage{tikz}
\usepackage{forest}

\lstdefinestyle{cppstyle}{
  language=C++,
  basicstyle=\ttfamily\small,
  keywordstyle=\color{blue}\bfseries,
  commentstyle=\color{gray},
  stringstyle=\color{orange},
  numbers=left,
  numberstyle=\tiny\color{gray},
  stepnumber=1,
  numbersep=5pt,
  frame=single,
  breaklines=true,
  showstringspaces=false,
  tabsize=2
}

\begin{document}

\title{資料結構與演算法入門:第 8 章}
\subtitle{樹 (Tree)}
\author{悠太翼 Yuuta Tsubasa}
\date{\today}

\frame{\titlepage}

\begin{frame}{什麼是樹?}
\begin{itemize}
    \item 樹是一種階層式的資料結構
    \item 由節點(Node)和邊(Edge)組成
    \item 有一個根節點(Root),其他節點形成子樹
    \item 沒有環路(Cycle)的連通圖
\end{itemize}

\vspace{1em}
\begin{block}{基本術語}
\begin{itemize}
    \item \textbf{根節點:}沒有父節點的節點
    \item \textbf{葉節點:}沒有子節點的節點
    \item \textbf{深度:}從根節點到某節點的邊數
    \item \textbf{高度:}從某節點到最深葉節點的邊數
    \item \textbf{度:}節點的子節點數量
\end{itemize}
\end{block}

\vspace{1em}
\begin{center}
\textbf{待補充:樹的結構圖解與術語標示}
\end{center}
\end{frame}

\begin{frame}{二元樹 (Binary Tree)}
\begin{itemize}
    \item 每個節點最多有兩個子節點
    \item 子節點分為左子節點和右子節點
    \item 是最常用的樹結構
\end{itemize}

\vspace{1em}
\begin{block}{特殊的二元樹}
\begin{itemize}
    \item \textbf{滿二元樹:}每個節點都有 0 或 2 個子節點
    \item \textbf{完全二元樹:}除了最後一層,其他層都填滿,最後一層從左到右填充
    \item \textbf{完美二元樹:}所有葉節點都在同一層,所有內部節點都有兩個子節點
\end{itemize}
\end{block}

\vspace{1em}
\textbf{性質:}
\begin{itemize}
    \item 高度為 $h$ 的二元樹最多有 $2^{h+1} - 1$ 個節點
    \item $n$ 個節點的完全二元樹高度為 $\lfloor \log_2 n \rfloor$
\end{itemize}
\end{frame}

\section{樹的遍歷}

\begin{frame}{前序遍歷 (Preorder Traversal)}
\begin{itemize}
    \item 訪問順序:根 → 左子樹 → 右子樹
    \item 遞迴定義:
    \begin{enumerate}
        \item 訪問根節點
        \item 前序遍歷左子樹
        \item 前序遍歷右子樹
    \end{enumerate}
\end{itemize}

\vspace{1em}
\begin{block}{應用}
\begin{itemize}
    \item 複製樹結構
    \item 建立表達式的前綴表示法
    \item 樹的序列化
\end{itemize}
\end{block}

\vspace{1em}
\textbf{時間複雜度:}$O(n)$\\
\textbf{空間複雜度:}$O(h)$,$h$ 為樹的高度

\vspace{1em}
\textbf{待補充:前序遍歷程式碼(遞迴與迭代版本)}
\end{frame}

\begin{frame}{中序遍歷 (Inorder Traversal)}
\begin{itemize}
    \item 訪問順序:左子樹 → 根 → 右子樹
    \item 遞迴定義:
    \begin{enumerate}
        \item 中序遍歷左子樹
        \item 訪問根節點
        \item 中序遍歷右子樹
    \end{enumerate}
\end{itemize}

\vspace{1em}
\begin{block}{特殊性質}
\begin{itemize}
    \item 對於二元搜尋樹,中序遍歷結果是有序的
    \item 可用於檢驗二元搜尋樹的正確性
\end{itemize}
\end{block}

\vspace{1em}
\textbf{應用:}
\begin{itemize}
    \item 表達式樹的中綴表示法
    \item 二元搜尋樹的排序輸出
\end{itemize}

\vspace{1em}
\textbf{待補充:中序遍歷程式碼與 BST 應用示例}
\end{frame}

\begin{frame}{後序遍歷 (Postorder Traversal)}
\begin{itemize}
    \item 訪問順序:左子樹 → 右子樹 → 根
    \item 遞迴定義:
    \begin{enumerate}
        \item 後序遍歷左子樹
        \item 後序遍歷右子樹
        \item 訪問根節點
    \end{enumerate}
\end{itemize}

\vspace{1em}
\begin{block}{應用}
\begin{itemize}
    \item 刪除樹的所有節點(從葉到根)
    \item 計算目錄大小
    \item 表達式樹的後綴表示法
    \item 計算樹的高度
\end{itemize}
\end{block}

\vspace{1em}
\textbf{特點:}在訪問節點之前,所有子節點都已經被訪問

\vspace{1em}
\textbf{待補充:後序遍歷程式碼與計算樹高範例}
\end{frame}

\section{廣度優先搜尋與深度優先搜尋}

\begin{frame}{廣度優先搜尋 (BFS)}
\begin{itemize}
    \item 按層級順序訪問節點
    \item 先訪問同一層的所有節點,再訪問下一層
    \item 使用佇列(Queue)實作
\end{itemize}

\vspace{1em}
\begin{block}{演算法步驟}
\begin{enumerate}
    \item 將根節點加入佇列
    \item 當佇列不為空時:
    \begin{itemize}
        \item 從佇列取出一個節點
        \item 訪問該節點
        \item 將其子節點加入佇列
    \end{itemize}
\end{enumerate}
\end{block}

\vspace{1em}
\textbf{應用:}
\begin{itemize}
    \item 層序遍歷
    \item 尋找最短路徑
    \item 檢測連通性
\end{itemize}

\vspace{1em}
\textbf{待補充:BFS 程式碼實作與視覺化}
\end{frame}

\begin{frame}{深度優先搜尋 (DFS)}
\begin{itemize}
    \item 先深入探索一條路徑,直到無法繼續為止
    \item 然後回溯,探索其他路徑
    \item 可以使用遞迴或堆疊實作
\end{itemize}

\vspace{1em}
\begin{block}{DFS vs 樹的遍歷}
\begin{itemize}
    \item 前序遍歷 = DFS 訪問節點的順序
    \item DFS 可以用於一般圖,不只是樹
    \item 可以檢測環路和連通性
\end{itemize}
\end{block}

\vspace{1em}
\textbf{應用:}
\begin{itemize}
    \item 拓撲排序
    \item 尋找路徑
    \item 連通性檢測
    \item 回溯演算法
\end{itemize}

\vspace{1em}
\textbf{待補充:DFS 遞迴與迭代實作}
\end{frame}

\section{二元搜尋樹}

\begin{frame}{二元搜尋樹 (Binary Search Tree, BST)}
\begin{itemize}
    \item 一種特殊的二元樹,滿足搜尋性質
    \item 對於任意節點:
    \begin{itemize}
        \item 左子樹的所有節點值 < 該節點值
        \item 右子樹的所有節點值 > 該節點值
    \end{itemize}
    \item 中序遍歷得到有序序列
\end{itemize}

\vspace{1em}
\begin{block}{基本操作}
\begin{itemize}
    \item \textbf{搜尋:}從根開始,比較並選擇左或右子樹
    \item \textbf{插入:}找到適當位置插入新節點
    \item \textbf{刪除:}三種情況(無子節點、一個子節點、兩個子節點)
\end{itemize}
\end{block}

\vspace{1em}
\textbf{平均時間複雜度:}$O(\log n)$\\
\textbf{最壞時間複雜度:}$O(n)$(退化為鏈表)
\end{frame}

\begin{frame}{BST 刪除操作}
\begin{itemize}
    \item BST 中最複雜的操作
    \item 需要保持 BST 性質
\end{itemize}

\vspace{1em}
\begin{block}{三種情況}
\begin{enumerate}
    \item \textbf{葉節點:}直接刪除
    \item \textbf{一個子節點:}用子節點替換
    \item \textbf{兩個子節點:}
    \begin{itemize}
        \item 找到右子樹的最小值(或左子樹的最大值)
        \item 用該值替換要刪除的節點值
        \item 刪除該最小值節點(遞迴)
    \end{itemize}
\end{enumerate}
\end{block}

\vspace{1em}
\textbf{待補充:BST 刪除操作圖解與程式碼}
\end{frame}

\section{平衡樹}

\begin{frame}{為什麼需要平衡樹?}
\begin{itemize}
    \item BST 在最壞情況下退化為鏈表,操作變成 $O(n)$
    \item 平衡樹透過旋轉操作維持平衡
    \item 保證操作的時間複雜度為 $O(\log n)$
\end{itemize}

\vspace{1em}
\begin{block}{平衡的定義}
\begin{itemize}
    \item 任意節點的左右子樹高度差不超過某個常數
    \item 不同平衡樹有不同的平衡條件
\end{itemize}
\end{block}

\vspace{1em}
\begin{center}
\textbf{常見的平衡樹:AVL 樹、紅黑樹}
\end{center}
\end{frame}

\begin{frame}{AVL 樹}
\begin{itemize}
    \item 最早的自平衡二元搜尋樹
    \item 平衡條件:任意節點的左右子樹高度差 ≤ 1
    \item 通過旋轉操作維持平衡
\end{itemize}

\vspace{1em}
\begin{block}{四種旋轉操作}
\begin{itemize}
    \item \textbf{LL 旋轉:}右旋
    \item \textbf{RR 旋轉:}左旋
    \item \textbf{LR 旋轉:}先左旋再右旋
    \item \textbf{RL 旋轉:}先右旋再左旋
\end{itemize}
\end{block}

\vspace{1em}
\textbf{時間複雜度:}所有操作 $O(\log n)$\\
\textbf{空間複雜度:}$O(n)$

\vspace{1em}
\textbf{待補充:AVL 樹旋轉操作圖解與實作}
\end{frame}

\begin{frame}{紅黑樹 (Red-Black Tree)}
\begin{itemize}
    \item 另一種自平衡二元搜尋樹
    \item 每個節點有顏色(紅或黑)
    \item 通過顏色規則維持近似平衡
\end{itemize}

\vspace{1em}
\begin{block}{紅黑樹性質}
\begin{enumerate}
    \item 每個節點不是紅色就是黑色
    \item 根節點是黑色
    \item 所有葉節點(NIL)是黑色
    \item 紅色節點的子節點必須是黑色
    \item 從任意節點到其葉節點的路徑包含相同數量的黑色節點
\end{enumerate}
\end{block}

\vspace{1em}
\textbf{優點:}插入和刪除的旋轉次數較 AVL 樹少\\
\textbf{應用:}C++ STL、Java TreeMap

\vspace{1em}
\textbf{待補充:紅黑樹性質說明與應用實例}
\end{frame}

\section{特殊樹結構}

\begin{frame}{線段樹 (Segment Tree)}
\begin{itemize}
    \item 用於解決區間查詢問題的樹結構
    \item 每個節點代表一個區間,儲存該區間的資訊
    \item 支援 $O(\log n)$ 的區間更新和查詢
\end{itemize}

\vspace{1em}
\begin{block}{常見應用}
\begin{itemize}
    \item 區間最小值查詢(RMQ)
    \item 區間和查詢
    \item 區間最大值查詢
    \item 懶惰傳播(Lazy Propagation)
\end{itemize}
\end{block}

\vspace{1em}
\textbf{建構時間:}$O(n)$\\
\textbf{查詢/更新時間:}$O(\log n)$\\
\textbf{空間複雜度:}$O(n)$

\vspace{1em}
\textbf{待補充:線段樹結構圖與基本操作實作}
\end{frame}

\begin{frame}{Fenwick 樹 (Binary Indexed Tree)}
\begin{itemize}
    \item 也稱為二進位索引樹
    \item 專門用於計算前綴和(Prefix Sum)
    \item 比線段樹更節省空間,實作更簡單
\end{itemize}

\vspace{1em}
\begin{block}{核心概念}
\begin{itemize}
    \item 利用二進位表示的性質
    \item 每個位置儲存特定長度的區間和
    \item 透過位元運算快速定位
\end{itemize}
\end{block}

\vspace{1em}
\textbf{lowbit 函數:}\texttt{x \& (-x)}\\
\textbf{更新:}$O(\log n)$\\
\textbf{查詢:}$O(\log n)$\\
\textbf{空間:}$O(n)$

\vspace{1em}
\textbf{待補充:Fenwick 樹原理解釋與程式碼實作}
\end{frame}

\section{樹的應用舉例}

\begin{frame}{舉例一:檔案系統}
\begin{itemize}
    \item 作業系統使用樹結構組織檔案和目錄
    \item 每個目錄是一個節點,檔案是葉節點
    \item 支援路徑導航和權限管理
\end{itemize}

\vspace{1em}
\textbf{操作:}
\begin{itemize}
    \item 建立目錄/檔案
    \item 刪除目錄/檔案
    \item 搜尋檔案
    \item 遍歷目錄結構
\end{itemize}

\vspace{1em}
\textbf{待補充:檔案系統樹結構圖解}
\end{frame}

\begin{frame}{舉例二:表達式樹}
\begin{itemize}
    \item 數學表達式的樹狀表示
    \item 運算子作為內部節點,運算元作為葉節點
    \item 可以用不同遍歷方式得到不同表示法
\end{itemize}

\vspace{1em}
\textbf{遍歷結果:}
\begin{itemize}
    \item 前序遍歷 → 前綴表示法
    \item 中序遍歷 → 中綴表示法
    \item 後序遍歷 → 後綴表示法
\end{itemize}

\vspace{1em}
\textbf{應用:}編譯器、計算器

\vspace{1em}
\textbf{待補充:表達式樹範例與不同表示法轉換}
\end{frame}

\begin{frame}{舉例三:決策樹}
\begin{itemize}
    \item 機器學習中的分類演算法
    \item 每個內部節點代表一個特徵測試
    \item 每個葉節點代表一個分類結果
    \item 從根到葉的路徑代表決策規則
\end{itemize}

\vspace{1em}
\textbf{優點:}
\begin{itemize}
    \item 易於理解和解釋
    \item 可以處理數值和類別特徵
    \item 不需要特徵正規化
\end{itemize}

\vspace{1em}
\textbf{待補充:決策樹範例與構建過程}
\end{frame}

\begin{frame}{本章總結}
\begin{itemize}
    \item \textbf{樹的遍歷:}前序、中序、後序、BFS、DFS
    \item \textbf{二元搜尋樹:}基本操作,刪除的複雜性
    \item \textbf{平衡樹:}
    \begin{itemize}
        \item AVL 樹:嚴格平衡,旋轉操作
        \item 紅黑樹:近似平衡,顏色規則
    \end{itemize}
    \item \textbf{特殊樹:}
    \begin{itemize}
        \item 線段樹:區間查詢問題
        \item Fenwick 樹:前綴和計算
    \end{itemize}
    \item 樹是許多高級資料結構和演算法的基礎
\end{itemize}

\vspace{1em}
\begin{center}
    \textbf{掌握樹結構,開啟高效演算法之門!}
\end{center}
\end{frame}

\end{document}