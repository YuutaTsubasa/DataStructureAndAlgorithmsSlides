\documentclass{beamer}
\usetheme{Madrid}

\usepackage{xeCJK}
\usepackage{fontspec}
\setmainfont[
  Path = fonts/,
  UprightFont = *-Regular,
  BoldFont = *-Bold,
  ItalicFont = *-Medium,
  BoldItalicFont = *-Bold
]{NotoSansTC}
\setCJKmainfont[
  Path = fonts/,
  UprightFont = *-Regular,
  BoldFont = *-Bold,
  ItalicFont = *-Medium,
  BoldItalicFont = *-Bold
]{NotoSansTC}

\usepackage{amsmath}
\usepackage{graphicx}
\usepackage{listings}
\usepackage{xcolor}

\lstdefinestyle{cppstyle}{
  language=C++,
  basicstyle=\ttfamily\small,
  keywordstyle=\color{blue}\bfseries,
  commentstyle=\color{gray},
  stringstyle=\color{orange},
  numbers=left,
  numberstyle=\tiny\color{gray},
  stepnumber=1,
  numbersep=5pt,
  frame=single,
  breaklines=true,
  showstringspaces=false,
  tabsize=2
}

\begin{document}

\title{資料結構與演算法入門:第 4 章}
\subtitle{貪婪法與動態規劃 (Greedy \& Dynamic Programming)}
\author{悠太翼 Yuuta Tsubasa}
\date{\today}

\frame{\titlepage}

\begin{frame}{什麼是貪婪法?}
\begin{itemize}
    \item 貪婪法(Greedy Algorithm)是一種演算法設計策略
    \item 在每一步都做出當下看起來最好的選擇
    \item 不會回頭修改之前的決定
    \item 希望透過局部最優解得到全域最優解
\end{itemize}

\vspace{1em}
\begin{block}{貪婪法的特點}
\begin{itemize}
    \item 簡單直觀,容易實作
    \item 執行效率通常很高
    \item 不一定能得到最優解
    \item 需要證明貪婪選擇的正確性
\end{itemize}
\end{block}
\end{frame}

\begin{frame}{什麼是動態規劃?}
\begin{itemize}
    \item 動態規劃(Dynamic Programming, DP)用於解決重疊子問題
    \item 將複雜問題分解成更小的子問題
    \item 儲存子問題的解,避免重複計算
    \item 適用於具有最優子結構的問題
\end{itemize}

\vspace{1em}
\begin{block}{動態規劃 vs 分而治之}
\begin{itemize}
    \item \textbf{分而治之:}子問題獨立,不重疊
    \item \textbf{動態規劃:}子問題重疊,需要記錄已計算的結果
\end{itemize}
\end{block}
\end{frame}

\section{Memorized Divide \& Conquer}

\begin{frame}{記憶化的分而治之}
\begin{itemize}
    \item 在分而治之基礎上加入記憶化(Memoization)
    \item 使用快取儲存已經計算過的子問題結果
    \item 避免重複計算,提升效率
    \item 是實現動態規劃的一種方法(由上而下)
\end{itemize}

\vspace{1em}
\begin{block}{經典例子:費波那契數列}
\begin{itemize}
    \item 原始遞迴:$T(n) = T(n-1) + T(n-2)$,時間複雜度 $O(2^n)$
    \item 加入記憶化:時間複雜度降為 $O(n)$
\end{itemize}
\end{block}

\vspace{1em}
\textbf{待補充:費波那契數列的記憶化實作}
\end{frame}

\section{經典動態規劃問題}

\begin{frame}{找零錢問題 (Coin Change Problem)}
\begin{itemize}
    \item \textbf{問題:}給定不同面額的硬幣和總金額,求最少硬幣數量
    \item \textbf{狀態定義:}$dp[i]$ 表示金額 $i$ 所需的最少硬幣數量
    \item \textbf{轉移方程:}$dp[i] = \min(dp[i-coin] + 1)$ 對所有硬幣面額
    \item \textbf{初始條件:}$dp[0] = 0$
\end{itemize}

\vspace{1em}
\textbf{範例:}硬幣面額 [1, 3, 4],總金額 6\\
\textbf{解:}6 = 3 + 3,需要 2 枚硬幣

\vspace{1em}
\textbf{待補充:程式碼實作與步驟圖解}
\end{frame}

\begin{frame}{最長遞增子序列 (Longest Increasing Subsequence, LIS)}
\begin{itemize}
    \item \textbf{問題:}找出陣列中最長的嚴格遞增子序列
    \item \textbf{狀態定義:}$dp[i]$ 表示以第 $i$ 個元素結尾的最長遞增子序列長度
    \item \textbf{轉移方程:}$dp[i] = \max(dp[j] + 1)$ 其中 $j < i$ 且 $arr[j] < arr[i]$
    \item \textbf{時間複雜度:}$O(n^2)$(可優化至 $O(n \log n)$)
\end{itemize}

\vspace{1em}
\textbf{範例:}[10, 9, 2, 5, 3, 7, 101, 18]\\
\textbf{解:}[2, 3, 7, 18] 或 [2, 3, 7, 101],長度為 4

\vspace{1em}
\textbf{待補充:程式碼實作與 $O(n \log n)$ 優化}
\end{frame}

\begin{frame}{最長共同子序列 (Longest Common Subsequence, LCS)}
\begin{itemize}
    \item \textbf{問題:}找出兩個序列的最長共同子序列
    \item \textbf{狀態定義:}$dp[i][j]$ 表示序列1前$i$個字符與序列2前$j$個字符的LCS長度
    \item \textbf{轉移方程:}
    \begin{itemize}
        \item 若 $s1[i-1] = s2[j-1]$:$dp[i][j] = dp[i-1][j-1] + 1$
        \item 否則:$dp[i][j] = \max(dp[i-1][j], dp[i][j-1])$
    \end{itemize}
    \item \textbf{時間複雜度:}$O(mn)$
\end{itemize}

\vspace{1em}
\textbf{範例:}"ABCDGH" 和 "AEDFHR"\\
\textbf{解:}"ADH",長度為 3

\vspace{1em}
\textbf{待補充:程式碼實作與表格填充過程}
\end{frame}

\begin{frame}{背包問題 (Knapsack Problem)}
\begin{itemize}
    \item \textbf{0-1背包:}每個物品只能選擇一次
    \item \textbf{狀態定義:}$dp[i][w]$ 表示前 $i$ 個物品、背包容量 $w$ 的最大價值
    \item \textbf{轉移方程:}
    \begin{itemize}
        \item 不選第 $i$ 個物品:$dp[i-1][w]$
        \item 選第 $i$ 個物品:$dp[i-1][w-weight[i]] + value[i]$
        \item $dp[i][w] = \max($兩者$)$
    \end{itemize}
    \item \textbf{時間複雜度:}$O(nW)$
\end{itemize}

\vspace{1em}
\textbf{完全背包:}每個物品可以選擇多次\\
\textbf{多重背包:}每個物品有數量限制

\vspace{1em}
\textbf{待補充:各種背包問題的程式碼實作}
\end{frame}

\section{動態規劃應用舉例}

\begin{frame}{舉例一:爬樓梯問題}
\begin{itemize}
    \item \textbf{問題:}爬 $n$ 階樓梯,每次可爬 1 或 2 階,有幾種方法?
    \item \textbf{狀態定義:}$dp[i]$ 表示爬到第 $i$ 階的方法數
    \item \textbf{轉移方程:}$dp[i] = dp[i-1] + dp[i-2]$
    \item \textbf{初始條件:}$dp[1] = 1$, $dp[2] = 2$
\end{itemize}

\vspace{1em}
\textbf{實際上就是費波那契數列!}

\vspace{1em}
\textbf{待補充:程式碼實作}
\end{frame}

\begin{frame}{舉例二:最小路徑和}
\begin{itemize}
    \item \textbf{問題:}在 $m \times n$ 網格中從左上角到右下角的最小路徑和
    \item \textbf{限制:}只能向右或向下移動
    \item \textbf{狀態定義:}$dp[i][j]$ 表示到達位置 $(i,j)$ 的最小路徑和
    \item \textbf{轉移方程:}$dp[i][j] = grid[i][j] + \min(dp[i-1][j], dp[i][j-1])$
\end{itemize}

\vspace{1em}
\textbf{可優化空間複雜度至 $O(n)$}

\vspace{1em}
\textbf{待補充:程式碼實作與空間優化}
\end{frame}

\begin{frame}{舉例三:編輯距離 (Edit Distance)}
\begin{itemize}
    \item \textbf{問題:}將字串 A 轉換成字串 B 所需的最少操作數
    \item \textbf{操作:}插入、刪除、替換字符
    \item \textbf{狀態定義:}$dp[i][j]$ 表示字串 A 前 $i$ 個字符轉換成字串 B 前 $j$ 個字符的最少操作數
    \item \textbf{轉移方程:}
    \begin{itemize}
        \item 字符相同:$dp[i][j] = dp[i-1][j-1]$
        \item 字符不同:$dp[i][j] = 1 + \min(dp[i-1][j], dp[i][j-1], dp[i-1][j-1])$
    \end{itemize}
\end{itemize}

\vspace{1em}
\textbf{待補充:程式碼實作與實際應用}
\end{frame}

\begin{frame}{貪婪法 vs 動態規劃}
\begin{center}
\renewcommand{\arraystretch}{1.4}
\begin{tabular}{|>{\centering\arraybackslash}m{2.5cm}|>{\raggedright\arraybackslash}m{4cm}|>{\raggedright\arraybackslash}m{4cm}|}
\hline
\textbf{比較項目} & \textbf{貪婪法} & \textbf{動態規劃} \\
\hline
決策方式 & 當下最優選擇 & 考慮所有可能的子問題 \\
\hline
時間複雜度 & 通常較低 & 可能較高,需要填表 \\
\hline
空間複雜度 & 通常 $O(1)$ & 需要額外儲存空間 \\
\hline
適用問題 & 具有貪婪選擇性質 & 具有最優子結構和重疊子問題 \\
\hline
正確性 & 需要證明 & 通過狀態轉移保證 \\
\hline
\end{tabular}
\end{center}
\end{frame}

\begin{frame}{本章總結}
\begin{itemize}
    \item \textbf{貪婪法:}每步做當下最優選擇,簡單高效但不一定最優
    \item \textbf{動態規劃:}解決重疊子問題,保證找到最優解
    \item \textbf{記憶化:}將分而治之與動態規劃結合的技巧
    \item \textbf{經典問題:}
    \begin{itemize}
        \item 找零錢問題
        \item 最長遞增子序列 (LIS)
        \item 最長共同子序列 (LCS)
        \item 背包問題
    \end{itemize}
    \item 選擇演算法策略需考慮問題特性和效率需求
\end{itemize}

\vspace{1em}
\begin{center}
    \textbf{掌握 DP,解決最優化問題的利器!}
\end{center}
\end{frame}

\end{document}