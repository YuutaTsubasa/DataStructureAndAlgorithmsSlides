\documentclass{beamer}
\usetheme{Madrid}

\usepackage{xeCJK}
\usepackage{fontspec}
\setmainfont[
  Path = fonts/,
  UprightFont = *-Regular,
  BoldFont = *-Bold,
  ItalicFont = *-Medium,
  BoldItalicFont = *-Bold
]{NotoSansTC}
\setCJKmainfont[
  Path = fonts/,
  UprightFont = *-Regular,
  BoldFont = *-Bold,
  ItalicFont = *-Medium,
  BoldItalicFont = *-Bold
]{NotoSansTC}

\usepackage{amsmath}
\usepackage{graphicx}
\usepackage{listings}
\usepackage{xcolor}
\usepackage{array}

\lstdefinestyle{cppstyle}{
  language=C++,
  basicstyle=\ttfamily\small,
  keywordstyle=\color{blue}\bfseries,
  commentstyle=\color{gray},
  stringstyle=\color{orange},
  numbers=left,
  numberstyle=\tiny\color{gray},
  stepnumber=1,
  numbersep=5pt,
  frame=single,
  breaklines=true,
  showstringspaces=false,
  tabsize=2
}

\begin{document}

\title{資料結構與演算法入門:第 7 章}
\subtitle{排序與搜尋 (Sort \& Search)}
\author{悠太翼 Yuuta Tsubasa}
\date{\today}

\frame{\titlepage}

\begin{frame}{為什麼要學習排序和搜尋?}
\begin{itemize}
    \item 排序和搜尋是電腦科學的基礎問題
    \item 幾乎所有應用都會用到
    \item 展示不同演算法設計技巧
    \item 效能差異巨大,影響實際應用
\end{itemize}

\vspace{1em}
\begin{block}{本章重點}
\begin{itemize}
    \item 8 種重要的排序演算法
    \item 2 種基本搜尋演算法
    \item 時間複雜度分析
    \item 穩定性和適用場景
\end{itemize}
\end{block}
\end{frame}

\section{簡單排序演算法}

\begin{frame}{氣泡排序 (Bubble Sort)}
\begin{itemize}
    \item 重複比較相鄰元素,如果順序錯誤就交換
    \item 每一輪會將最大值「浮」到最後
    \item 名稱來源:小元素像氣泡一樣上浮
\end{itemize}

\vspace{1em}
\begin{block}{演算法步驟}
\begin{enumerate}
    \item 比較相鄰的元素
    \item 如果第一個比第二個大,就交換它們
    \item 對每一對相鄰元素做同樣的工作
    \item 重複以上步驟,直到沒有需要交換的元素
\end{enumerate}
\end{block}

\vspace{1em}
\textbf{時間複雜度:}$O(n^2)$\\
\textbf{空間複雜度:}$O(1)$\\
\textbf{穩定性:}穩定

\vspace{1em}
\textbf{待補充:氣泡排序動畫演示與程式碼}
\end{frame}

\begin{frame}{插入排序 (Insertion Sort)}
\begin{itemize}
    \item 將陣列分為已排序和未排序兩部分
    \item 每次從未排序部分取一個元素,插入到已排序部分的正確位置
    \item 就像整理撲克牌的過程
\end{itemize}

\vspace{1em}
\begin{block}{演算法特點}
\begin{itemize}
    \item 對於小規模資料表現良好
    \item 對於近乎排序的資料效率很高
    \item 是自適應演算法(adaptive)
\end{itemize}
\end{block}

\vspace{1em}
\textbf{時間複雜度:}最佳 $O(n)$,平均/最壞 $O(n^2)$\\
\textbf{空間複雜度:}$O(1)$\\
\textbf{穩定性:}穩定

\vspace{1em}
\textbf{待補充:插入排序動畫演示與程式碼}
\end{frame}

\begin{frame}{選擇排序 (Selection Sort)}
\begin{itemize}
    \item 每次從未排序部分找出最小值
    \item 將最小值與未排序部分的第一個元素交換
    \item 重複此過程直到全部排序完成
\end{itemize}

\vspace{1em}
\begin{block}{演算法特點}
\begin{itemize}
    \item 交換次數最少(最多 $n-1$ 次)
    \item 不受資料初始順序影響
    \item 實作簡單
\end{itemize}
\end{block}

\vspace{1em}
\textbf{時間複雜度:}$O(n^2)$(所有情況)\\
\textbf{空間複雜度:}$O(1)$\\
\textbf{穩定性:}不穩定

\vspace{1em}
\textbf{待補充:選擇排序動畫演示與程式碼}
\end{frame}

\section{高效排序演算法}

\begin{frame}{快速排序 (Quick Sort)}
\begin{itemize}
    \item 分而治之的經典應用
    \item 選擇一個基準點(pivot),將陣列分為小於和大於基準的兩部分
    \item 遞迴地對兩部分進行排序
\end{itemize}

\vspace{1em}
\begin{block}{Partition 步驟}
\begin{enumerate}
    \item 選擇基準點(通常是最後一個元素)
    \item 將小於基準的元素移到左邊
    \item 將大於基準的元素移到右邊
    \item 將基準點放到正確位置
\end{enumerate}
\end{block}

\vspace{1em}
\textbf{時間複雜度:}最佳/平均 $O(n \log n)$,最壞 $O(n^2)$\\
\textbf{空間複雜度:}$O(\log n)$\\
\textbf{穩定性:}不穩定

\vspace{1em}
\textbf{待補充:快速排序詳細演示與優化技巧}
\end{frame}

\begin{frame}{合併排序 (Merge Sort)}
\begin{itemize}
    \item 分而治之策略,保證穩定的 $O(n \log n)$ 效能
    \item 將陣列分為兩半,遞迴排序,然後合併
    \item 合併步驟是關鍵:將兩個已排序陣列合併成一個
\end{itemize}

\vspace{1em}
\begin{block}{演算法優點}
\begin{itemize}
    \item 時間複雜度穩定
    \item 穩定排序
    \item 適合外部排序(資料量超過記憶體)
\end{itemize}
\end{block}

\vspace{1em}
\textbf{時間複雜度:}$O(n \log n)$(所有情況)\\
\textbf{空間複雜度:}$O(n)$\\
\textbf{穩定性:}穩定

\vspace{1em}
\textbf{待補充:合併排序演示與合併過程詳解}
\end{frame}

\section{特殊排序演算法}

\begin{frame}{桶排序 (Bucket Sort)}
\begin{itemize}
    \item 將元素分配到有限數量的桶中
    \item 每個桶內部進行排序
    \item 按順序串接所有桶的結果
\end{itemize}

\vspace{1em}
\begin{block}{適用條件}
\begin{itemize}
    \item 資料分布均勻
    \item 知道資料的範圍
    \item 可以設計適當的雜湊函數
\end{itemize}
\end{block}

\vspace{1em}
\textbf{時間複雜度:}平均 $O(n+k)$,最壞 $O(n^2)$\\
\textbf{空間複雜度:}$O(n+k)$\\
\textbf{穩定性:}取決於桶內排序演算法

\vspace{1em}
\textbf{待補充:桶排序圖解與實作}
\end{frame}

\begin{frame}{計數排序 (Counting Sort)}
\begin{itemize}
    \item 不基於比較的排序演算法
    \item 統計每個值出現的次數
    \item 根據計數結果重建排序陣列
\end{itemize}

\vspace{1em}
\begin{block}{演算法步驟}
\begin{enumerate}
    \item 建立計數陣列,大小為值域範圍
    \item 統計每個值的出現次數
    \item 計算累積次數
    \item 根據累積次數放置元素
\end{enumerate}
\end{block}

\vspace{1em}
\textbf{時間複雜度:}$O(n+k)$,$k$ 為值域範圍\\
\textbf{空間複雜度:}$O(k)$\\
\textbf{穩定性:}穩定

\vspace{1em}
\textbf{待補充:計數排序詳細步驟與程式碼}
\end{frame}

\begin{frame}{基數排序 (Radix Sort)}
\begin{itemize}
    \item 按照數字的位數進行排序
    \item 從最低位開始,逐位進行穩定排序
    \item 通常使用計數排序作為子排序演算法
\end{itemize}

\vspace{1em}
\begin{block}{LSD vs MSD}
\begin{itemize}
    \item \textbf{LSD (Least Significant Digit):}從個位數開始
    \item \textbf{MSD (Most Significant Digit):}從最高位開始
\end{itemize}
\end{block}

\vspace{1em}
\textbf{時間複雜度:}$O(d \times (n+k))$,$d$ 為最大位數\\
\textbf{空間複雜度:}$O(n+k)$\\
\textbf{穩定性:}穩定

\vspace{1em}
\textbf{待補充:基數排序演示與不同進制的處理}
\end{frame}

\section{搜尋演算法}

\begin{frame}{線性搜尋 (Linear Search)}
\begin{itemize}
    \item 最簡單的搜尋演算法
    \item 從頭到尾逐一檢查每個元素
    \item 適用於未排序的資料
\end{itemize}

\vspace{1em}
\begin{block}{演算法特點}
\begin{itemize}
    \item 不需要資料預處理
    \item 適用於任何資料結構
    \item 實作簡單
\end{itemize}
\end{block}

\vspace{1em}
\textbf{時間複雜度:}最佳 $O(1)$,平均/最壞 $O(n)$\\
\textbf{空間複雜度:}$O(1)$

\vspace{1em}
\textbf{變形:}
\begin{itemize}
    \item Sentinel Search(哨兵搜尋)
    \item 雙向線性搜尋
\end{itemize}

\vspace{1em}
\textbf{待補充:線性搜尋程式碼與優化}
\end{frame}

\begin{frame}{二分搜尋 (Binary Search)}
\begin{itemize}
    \item 在已排序陣列中搜尋特定元素
    \item 每次比較中間元素,排除一半搜尋空間
    \item 分而治之的經典應用
\end{itemize}

\vspace{1em}
\begin{block}{演算法步驟}
\begin{enumerate}
    \item 比較目標值與中間元素
    \item 如果相等,找到目標
    \item 如果目標值較小,搜尋左半部
    \item 如果目標值較大,搜尋右半部
    \item 重複直到找到或搜尋空間為空
\end{enumerate}
\end{block}

\vspace{1em}
\textbf{時間複雜度:}$O(\log n)$\\
\textbf{空間複雜度:}$O(1)$ 迭代版本,$O(\log n)$ 遞迴版本

\vspace{1em}
\textbf{前提條件:}資料必須已排序
\end{frame}

\begin{frame}{Lower Bound 與 Upper Bound}
\begin{itemize}
    \item 在有重複元素的已排序陣列中的進階搜尋
    \item \textbf{Lower Bound:}第一個大於等於目標值的位置
    \item \textbf{Upper Bound:}第一個大於目標值的位置
\end{itemize}

\vspace{1em}
\begin{block}{應用場景}
\begin{itemize}
    \item 計算目標值出現次數
    \item 範圍查詢
    \item 插入位置確定
\end{itemize}
\end{block}

\vspace{1em}
\textbf{範例:}陣列 [1, 2, 2, 2, 3, 4],目標值 2\\
\textbf{Lower Bound:}index 1\\
\textbf{Upper Bound:}index 4\\
\textbf{出現次數:}4 - 1 = 3

\vspace{1em}
\textbf{待補充:Lower/Upper Bound 實作與應用}
\end{frame}

\begin{frame}{排序演算法比較}
\begin{center}
\renewcommand{\arraystretch}{1.2}
\scriptsize
\begin{tabular}{|>{\centering\arraybackslash}m{1.8cm}|>{\centering\arraybackslash}m{1.5cm}|>{\centering\arraybackslash}m{1.5cm}|>{\centering\arraybackslash}m{1.2cm}|>{\centering\arraybackslash}m{1cm}|}
\hline
\textbf{演算法} & \textbf{平均時間} & \textbf{最壞時間} & \textbf{空間複雜度} & \textbf{穩定} \\
\hline
Bubble Sort & $O(n^2)$ & $O(n^2)$ & $O(1)$ & 是 \\
\hline
Insertion Sort & $O(n^2)$ & $O(n^2)$ & $O(1)$ & 是 \\
\hline
Selection Sort & $O(n^2)$ & $O(n^2)$ & $O(1)$ & 否 \\
\hline
Quick Sort & $O(n \log n)$ & $O(n^2)$ & $O(\log n)$ & 否 \\
\hline
Merge Sort & $O(n \log n)$ & $O(n \log n)$ & $O(n)$ & 是 \\
\hline
Bucket Sort & $O(n+k)$ & $O(n^2)$ & $O(n+k)$ & 是 \\
\hline
Counting Sort & $O(n+k)$ & $O(n+k)$ & $O(k)$ & 是 \\
\hline
Radix Sort & $O(d(n+k))$ & $O(d(n+k))$ & $O(n+k)$ & 是 \\
\hline
\end{tabular}
\end{center}

\vspace{1em}
\textbf{選擇建議:}
\begin{itemize}
    \item 一般用途:Quick Sort 或 Merge Sort
    \item 需要穩定:Merge Sort
    \item 整數且範圍小:Counting Sort
    \item 小規模資料:Insertion Sort
\end{itemize}
\end{frame}

\begin{frame}{本章總結}
\begin{itemize}
    \item \textbf{排序演算法分類:}
    \begin{itemize}
        \item 簡單排序:Bubble、Insertion、Selection
        \item 高效排序:Quick、Merge
        \item 特殊排序:Bucket、Counting、Radix
    \end{itemize}
    \item \textbf{搜尋演算法:}
    \begin{itemize}
        \item Linear Search:適用於未排序資料
        \item Binary Search:適用於已排序資料
    \end{itemize}
    \item \textbf{選擇考量:}時間複雜度、空間複雜度、穩定性、適用場景
    \item 排序是許多其他演算法的基礎
\end{itemize}

\vspace{1em}
\begin{center}
    \textbf{掌握排序搜尋,奠定演算法基礎!}
\end{center}
\end{frame}

\end{document}