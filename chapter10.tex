\documentclass{beamer}
\usetheme{Madrid}

\usepackage{xeCJK}
\usepackage{fontspec}
\setmainfont[
  Path = fonts/,
  UprightFont = *-Regular,
  BoldFont = *-Bold,
  ItalicFont = *-Medium,
  BoldItalicFont = *-Bold
]{NotoSansTC}
\setCJKmainfont[
  Path = fonts/,
  UprightFont = *-Regular,
  BoldFont = *-Bold,
  ItalicFont = *-Medium,
  BoldItalicFont = *-Bold
]{NotoSansTC}

\usepackage{amsmath}
\usepackage{graphicx}
\usepackage{listings}
\usepackage{xcolor}
\usepackage{tikz}
\usepackage{array}

\lstdefinestyle{cppstyle}{
  language=C++,
  basicstyle=\ttfamily\small,
  keywordstyle=\color{blue}\bfseries,
  commentstyle=\color{gray},
  stringstyle=\color{orange},
  numbers=left,
  numberstyle=\tiny\color{gray},
  stepnumber=1,
  numbersep=5pt,
  frame=single,
  breaklines=true,
  showstringspaces=false,
  tabsize=2
}

\begin{document}

\title{資料結構與演算法入門:第 10 章}
\subtitle{圖 (Graph)}
\author{悠太翼 Yuuta Tsubasa}
\date{\today}

\frame{\titlepage}

\begin{frame}{什麼是圖?}
\begin{itemize}
    \item 圖(Graph)是由頂點(Vertex)和邊(Edge)組成的資料結構
    \item 表示為 $G = (V, E)$,其中 $V$ 是頂點集合,$E$ 是邊集合
    \item 比樹更一般化的結構,可以有環路
    \item 用於建模各種關係和網路
\end{itemize}

\vspace{1em}
\begin{block}{基本術語}
\begin{itemize}
    \item \textbf{相鄰(Adjacent):}兩頂點間有邊相連
    \item \textbf{度數(Degree):}與頂點相連的邊數
    \item \textbf{路徑(Path):}連接兩頂點的頂點序列
    \item \textbf{環(Cycle):}起點和終點相同的路徑
    \item \textbf{連通(Connected):}任意兩頂點間都有路徑
\end{itemize>
\end{block>

\vspace{1em}
\begin{center>
\textbf{待補充:圖的基本概念圖解}
\end{center>
\end{frame>

\begin{frame}{圖的分類}
\begin{itemize>
    \item \textbf{有向圖 vs 無向圖:}邊是否有方向
    \item \textbf{加權圖 vs 非加權圖:}邊是否有權重
    \item \textbf{簡單圖 vs 多重圖:}是否允許重複邊和自環
    \item \textbf{稠密圖 vs 稀疏圖:}邊數相對於頂點數的多寡
\end{itemize>

\vspace{1em}
\begin{center>
\renewcommand{\arraystretch}{1.3}
\begin{tabular}{|>{\centering\arraybackslash}m{3cm}|>{\centering\arraybackslash}m{3cm}|>{\centering\arraybackslash}m{3cm}|}
\hline
\textbf{圖類型} & \textbf{邊數範圍} & \textbf{適用表示法} \\
\hline
稀疏圖 & $|E| = O(|V|)$ & 鄰接串列 \\
\hline
稠密圖 & $|E| = O(|V|^2)$ & 鄰接矩陣 \\
\hline
\end{tabular>
\end{center>

\vspace{1em}
\textbf{實際例子:}
\begin{itemize>
    \item 社交網路(有向、無權重)
    \item 道路網路(無向、有權重)
    \item 網際網路(有向、可能有權重)
\end{itemize>
\end{frame>

\begin{frame}{圖的表示法}
\begin{itemize>
    \item \textbf{鄰接矩陣(Adjacency Matrix):}二維陣列表示
    \item \textbf{鄰接串列(Adjacency List):}每個頂點維護相鄰頂點的串列
    \item \textbf{邊串列(Edge List):}直接儲存所有邊的資訊
\end{itemize>

\vspace{1em}
\begin{center>
\renewcommand{\arraystretch}{1.3}
\begin{tabular}{|>{\centering\arraybackslash}m{2.5cm}|>{\centering\arraybackslash}m{2cm}|>{\centering\arraybackslash}m{2cm}|>{\centering\arraybackslash}m{2cm}|}
\hline
\textbf{表示法} & \textbf{空間複雜度} & \textbf{查詢邊} & \textbf{遍歷鄰居} \\
\hline
鄰接矩陣 & $O(V^2)$ & $O(1)$ & $O(V)$ \\
\hline
鄰接串列 & $O(V+E)$ & $O(V)$ & $O(degree)$ \\
\hline
邊串列 & $O(E)$ & $O(E)$ & $O(E)$ \\
\hline
\end{tabular>
\end{center>

\vspace{1em}
\textbf{待補充:三種表示法的圖解與程式碼實作}
\end{frame>

\section{最小生成樹}

\begin{frame}{什麼是最小生成樹?}
\begin{itemize>
    \item 最小生成樹(Minimum Spanning Tree, MST)是連通加權無向圖的一個子圖
    \item 包含所有頂點,且邊的權重總和最小
    \item 恰好有 $|V| - 1$ 條邊,形成樹結構
    \item 應用:網路設計、聚類分析
\end{itemize>

\vspace{1em}
\begin{block}{MST 的性質}
\begin{itemize>
    \item \textbf{切割性質:}對於任意切割,權重最小的跨越邊必在某個 MST 中
    \item \textbf{環路性質:}對於任意環路,權重最大的邊不在任何 MST 中
    \item \textbf{唯一性:}如果所有邊權重都不同,MST 是唯一的
\end{itemize>
\end{block>

\vspace{1em}
\textbf{經典演算法:}Prim's Algorithm、Kruskal's Algorithm

\vspace{1em}
\textbf{待補充:MST 概念圖解與性質說明}
\end{frame>

\begin{frame}{Prim's Algorithm}
\begin{itemize>
    \item 從任意頂點開始,逐步擴展 MST
    \item 每次加入權重最小的跨越邊
    \item 使用優先佇列選擇最小權重邊
\end{itemize>

\vspace{1em}
\begin{block}{演算法步驟}
\begin{enumerate>
    \item 選擇任意起始頂點,加入 MST
    \item 將起始頂點的所有邊加入優先佇列
    \item 重複以下步驟直到包含所有頂點:
    \begin{itemize>
        \item 從優先佇列取出權重最小的邊
        \item 如果該邊連接到新頂點,加入 MST
        \item 將新頂點的所有邊加入優先佇列
    \end{itemize>
\end{enumerate>
\end{block>

\vspace{1em}
\textbf{時間複雜度:}$O(E \log V)$(使用二元堆積)\\
\textbf{空間複雜度:}$O(V)$

\vspace{1em}
\textbf{待補充:Prim's Algorithm 步驟演示與程式碼}
\end{frame>

\begin{frame}{Kruskal's Algorithm}
\begin{itemize>
    \item 按邊的權重排序,逐條考慮是否加入 MST
    \item 使用並查集(Union-Find)檢測環路
    \item 貪婪策略:每次選擇權重最小且不形成環路的邊
\end{itemize>

\vspace{1em}
\begin{block}{演算法步驟}
\begin{enumerate>
    \item 將所有邊按權重排序
    \item 初始化並查集,每個頂點為獨立集合
    \item 依序考慮每條邊:
    \begin{itemize>
        \item 如果邊的兩端點不在同一集合,加入 MST
        \item 合併兩端點所在的集合
    \end{itemize>
    \item 直到 MST 包含 $|V| - 1$ 條邊
\end{enumerate>
\end{block>

\vspace{1em}
\textbf{時間複雜度:}$O(E \log E)$(主要是排序)\\
\textbf{空間複雜度:}$O(V)$

\vspace{1em}
\textbf{待補充:Kruskal's Algorithm 演示與並查集應用}
\end{frame>

\section{最短路徑}

\begin{frame}{最短路徑問題}
\begin{itemize>
    \item 在加權圖中尋找兩點間權重和最小的路徑
    \item 根據問題類型分為不同變形:
    \begin{itemize>
        \item 單源最短路徑:從一個頂點到所有其他頂點
        \item 單對最短路徑:兩個特定頂點間
        \item 全對最短路徑:所有頂點對之間
    \end{itemize>
\end{itemize>

\vspace{1em}
\begin{block}{演算法選擇}
\begin{itemize>
    \item \textbf{非負權重:}Dijkstra's Algorithm
    \item \textbf{有負權重:}Bellman-Ford Algorithm
    \item \textbf{全對最短路徑:}Floyd-Warshall Algorithm
\end{itemize>
\end{block>

\vspace{1em}
\textbf{應用:}導航系統、網路路由、遊戲 AI
\end{frame>

\begin{frame}{Dijkstra's Algorithm}
\begin{itemize>
    \item 解決單源最短路徑問題(非負權重)
    \item 貪婪策略:每次選擇距離最小的未訪問頂點
    \item 使用優先佇列提高效率
\end{itemize>

\vspace{1em}
\begin{block}{演算法步驟}
\begin{enumerate>
    \item 初始化所有頂點距離為無窮大,起點距離為 0
    \item 將所有頂點加入優先佇列
    \item 重複以下步驟直到佇列為空:
    \begin{itemize>
        \item 取出距離最小的頂點 $u$
        \item 對 $u$ 的每個鄰居 $v$,檢查是否能更新最短距離
        \item 如果 $dist[u] + weight(u,v) < dist[v]$,更新 $dist[v]$
    \end{itemize>
\end{enumerate>
\end{block>

\vspace{1em}
\textbf{時間複雜度:}$O((V + E) \log V)$(使用二元堆積)\\
\textbf{正確性:}基於最優子結構性質

\vspace{1em}
\textbf{待補充:Dijkstra 演算法詳細演示}
\end{frame>

\begin{frame}{Bellman-Ford Algorithm}
\begin{itemize>
    \item 解決單源最短路徑問題(允許負權重邊)
    \item 可以檢測負權重環路
    \item 使用動態規劃的思想
\end{itemize>

\vspace{1em}
\begin{block}{演算法步驟}
\begin{enumerate>
    \item 初始化所有頂點距離為無窮大,起點距離為 0
    \item 重複 $|V| - 1$ 次:
    \begin{itemize>
        \item 對每條邊 $(u, v)$ 進行鬆弛操作
        \item 如果 $dist[u] + weight(u,v) < dist[v]$,更新 $dist[v]$
    \end{itemize>
    \item 檢查負權重環路:再次檢查所有邊,如果還能更新距離,則有負環路
\end{enumerate>
\end{block>

\vspace{1em}
\textbf{時間複雜度:}$O(VE)$\\
\textbf{空間複雜度:}$O(V)$

\vspace{1em}
\textbf{待補充:Bellman-Ford 演算法與負環路檢測}
\end{frame>

\section{網路流}

\begin{frame}{最大流問題 (Maximum Flow)}
\begin{itemize>
    \item 在流量網路中尋找從源點到匯點的最大流量
    \item 每條邊有容量限制
    \item 滿足流量守恆:除了源點和匯點,每個頂點的入流等於出流
\end{itemize>

\vspace{1em}
\begin{block}{基本概念}
\begin{itemize>
    \item \textbf{容量(Capacity):}邊能承載的最大流量
    \item \textbf{流量(Flow):}邊上實際的流量
    \item \textbf{殘餘網路(Residual Network):}剩餘容量形成的網路
    \item \textbf{增廣路徑(Augmenting Path):}從源點到匯點且有剩餘容量的路徑
\end{itemize>
\end{block>

\vspace{1em}
\textbf{最大流最小切定理:}最大流值等於最小切的容量

\vspace{1em}
\textbf{應用:}網路頻寬分配、運輸問題、二分圖匹配
\end{frame>

\begin{frame}{Ford-Fulkerson Algorithm}
\begin{itemize>
    \item 最大流問題的經典演算法
    \item 重複尋找增廣路徑,直到沒有增廣路徑為止
    \item 具體實作取決於尋找增廣路徑的方法
\end{itemize>

\vspace{1em}
\begin{block}{演算法框架}
\begin{enumerate>
    \item 初始化所有邊的流量為 0
    \item 重複以下步驟直到沒有增廣路徑:
    \begin{itemize>
        \item 在殘餘網路中尋找從源點到匯點的路徑
        \item 沿路徑增加流量(bottleneck capacity)
        \item 更新殘餘網路
    \end{itemize>
    \item 返回最大流值
\end{enumerate>
\end{block>

\vspace{1em}
\textbf{變形:}
\begin{itemize>
    \item Edmonds-Karp:使用 BFS 尋找增廣路徑,$O(VE^2)$
    \item Dinic's Algorithm:使用層次圖,$O(V^2E)$
\end{itemize>

\vspace{1em}
\textbf{待補充:Ford-Fulkerson 演算法演示}
\end{frame>

\section{圖的應用舉例}

\begin{frame}{舉例一:社交網路分析}
\begin{itemize>
    \item 頂點代表用戶,邊代表朋友關係
    \item 分析社群結構、影響力傳播、推薦系統
\end{itemize>

\vspace{1em}
\begin{block}{相關演算法}
\begin{itemize}
    \item \textbf{中心性測量:}找出重要節點
    \item \textbf{社群檢測:}發現緊密連接的群組
    \item \textbf{PageRank:}評估網頁/用戶重要性
    \item \textbf{六度分離:}計算用戶間的距離
\end{itemize>
\end{block>

\vspace{1em}
\textbf{待補充:社交網路圖範例與分析方法}
\end{frame>

\begin{frame}{舉例二:交通路網規劃}
\begin{itemize>
    \item 頂點代表地點,邊代表道路,權重代表距離或時間
    \item 導航系統的核心技術
\end{itemize>

\vspace{1em}
\begin{block}{實際考量}
\begin{itemize>
    \item \textbf{即時交通:}動態更新邊的權重
    \item \textbf{多重條件:}時間、距離、費用的權衡
    \item \textbf{限制條件:}車輛類型、道路限制
    \item \textbf{效率最佳化:}預處理、分層圖
\end{itemize>
\end{block>

\vspace{1em}
\textbf{進階演算法:}
\begin{itemize>
    \item A* 搜尋(啟發式)
    \item Contraction Hierarchies
    \item Hub Labeling
\end{itemize>

\vspace{1em}
\textbf{待補充:導航系統演算法原理}
\end{frame>

\begin{frame}{舉例三:網際網路路由}
\begin{itemize>
    \item 路由器作為頂點,連線作為邊
    \item 封包在網路中的路徑選擇
\end{itemize>

\vspace{1em}
\begin{block}{路由演算法}
\begin{itemize>
    \item \textbf{距離向量:}基於 Bellman-Ford
    \item \textbf{鏈路狀態:}基於 Dijkstra
    \item \textbf{BGP:}邊界閘道協定,處理自治系統間路由
\end{itemize>
\end{block>

\vspace{1em}
\textbf{挑戰:}
\begin{itemize}
    \item 網路拓撲動態變化
    \item 大規模網路的可擴展性
    \item 安全性和可靠性
\end{itemize>

\vspace{1em}
\textbf{待補充:網際網路路由協定原理}
\end{frame>

\begin{frame}{本章總結}
\begin{itemize>
    \item \textbf{圖的基礎:}頂點、邊、表示法(鄰接矩陣、鄰接串列)
    \item \textbf{最小生成樹:}
    \begin{itemize}
        \item Prim's Algorithm:逐步擴展
        \item Kruskal's Algorithm:邊排序 + 並查集
    \end{itemize>
    \item \textbf{最短路徑:}
    \begin{itemize>
        \item Dijkstra:非負權重,單源
        \item Bellman-Ford:負權重,負環路檢測
    \end{itemize>
    \item \textbf{網路流:}Ford-Fulkerson、最大流最小切
    \item \textbf{應用廣泛:}社交網路、交通、網際網路、生物資訊
\end{itemize>

\vspace{1em}
\begin{center}
    \textbf{圖論是解決複雜關係問題的強大工具!}
\end{center>
\end{frame>

\end{document>