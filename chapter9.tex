\documentclass{beamer}
\usetheme{Madrid}

\usepackage{xeCJK}
\usepackage{fontspec}
\setmainfont[
  Path = fonts/,
  UprightFont = *-Regular,
  BoldFont = *-Bold,
  ItalicFont = *-Medium,
  BoldItalicFont = *-Bold
]{NotoSansTC}
\setCJKmainfont[
  Path = fonts/,
  UprightFont = *-Regular,
  BoldFont = *-Bold,
  ItalicFont = *-Medium,
  BoldItalicFont = *-Bold
]{NotoSansTC}

\usepackage{amsmath}
\usepackage{graphicx}
\usepackage{listings}
\usepackage{xcolor}
\usepackage{tikz}
\usepackage{array}

\lstdefinestyle{cppstyle}{
  language=C++,
  basicstyle=\ttfamily\small,
  keywordstyle=\color{blue}\bfseries,
  commentstyle=\color{gray},
  stringstyle=\color{orange},
  numbers=left,
  numberstyle=\tiny\color{gray},
  stepnumber=1,
  numbersep=5pt,
  frame=single,
  breaklines=true,
  showstringspaces=false,
  tabsize=2
}

\begin{document}

\title{資料結構與演算法入門:第 9 章}
\subtitle{堆積 (Heap)}
\author{悠太翼 Yuuta Tsubasa}
\date{\today}

\frame{\titlepage}

\begin{frame}{什麼是堆積?}
\begin{itemize}
    \item 堆積(Heap)是一種特殊的完全二元樹
    \item 滿足堆積性質(Heap Property)
    \item 通常用陣列實作,節省指標空間
    \item 是優先佇列(Priority Queue)的最佳實作方式
\end{itemize}

\vspace{1em}
\begin{block}{堆積性質}
\begin{itemize}
    \item \textbf{最大堆積(Max Heap):}父節點的值 $\geq$ 子節點的值
    \item \textbf{最小堆積(Min Heap):}父節點的值 $\leq$ 子節點的值
\end{itemize}
\end{block}

\vspace{1em}
\textbf{注意:}堆積不是二元搜尋樹!只保證父子關係,不保證左右關係

\vspace{1em}
\begin{center}
\textbf{待補充:最大堆積與最小堆積圖解}
\end{center}
\end{frame}

\begin{frame}{堆積的陣列表示法}
\begin{itemize}
    \item 使用陣列儲存完全二元樹
    \item 節點與陣列索引的對應關係(從索引 1 開始):
    \begin{itemize}
        \item 父節點:$\lfloor i/2 \rfloor$
        \item 左子節點:$2i$
        \item 右子節點:$2i + 1$
    \end{itemize}
    \item 根節點在索引 1,索引 0 可以作為哨兵
\end{itemize}

\vspace{1em}
\begin{block}{優點}
\begin{itemize}
    \item 不需要額外的指標空間
    \item 快速計算父子關係
    \item 記憶體存取具有良好的局部性
\end{itemize}
\end{block}

\vspace{1em}
\textbf{待補充:陣列表示法圖解與索引計算示例}
\end{frame}

\begin{frame}{堆積的基本操作}
\begin{itemize}
    \item \textbf{插入(Insert):}在堆積末端加入新元素,然後上浮
    \item \textbf{刪除最值(Extract-Max/Min):}移除根節點,重新調整堆積
    \item \textbf{查看最值(Peek):}返回根節點的值
    \item \textbf{建堆(Build Heap):}將任意陣列轉換成堆積
\end{itemize}

\vspace{1em}
\begin{block}{輔助操作}
\begin{itemize}
    \item \textbf{上浮(Heapify Up):}向上調整維持堆積性質
    \item \textbf{下沉(Heapify Down):}向下調整維持堆積性質
\end{itemize}
\end{block}

\vspace{1em}
\textbf{時間複雜度:}
\begin{itemize}
    \item 插入、刪除:$O(\log n)$
    \item 查看最值:$O(1)$
    \item 建堆:$O(n)$
\end{itemize}
\end{frame}

\begin{frame}{插入操作 (Insert)}
\begin{itemize}
    \item 將新元素加入陣列末端(堆積的最後位置)
    \item 與父節點比較,如果違反堆積性質就交換
    \item 重複此過程直到滿足堆積性質或到達根節點
\end{itemize}

\vspace{1em}
\begin{block}{上浮(Heapify Up)演算法}
\begin{enumerate}
    \item 從新插入的節點開始
    \item 與父節點比較
    \item 如果違反堆積性質,交換位置
    \item 繼續向上檢查,直到滿足條件
\end{enumerate}
\end{block}

\vspace{1em}
\textbf{最壞情況:}新元素需要移動到根節點,需要 $O(\log n)$ 時間

\vspace{1em}
\textbf{待補充:插入操作圖解與程式碼實作}
\end{frame}

\begin{frame}{刪除最值操作 (Extract-Max/Min)}
\begin{itemize}
    \item 移除根節點(最大值或最小值)
    \item 將最後一個元素移到根節點位置
    \item 透過下沉操作重新調整堆積
\end{itemize}

\vspace{1em}
\begin{block}{下沉(Heapify Down)演算法}
\begin{enumerate}
    \item 從根節點開始
    \item 與子節點比較,找出最大/最小的子節點
    \item 如果違反堆積性質,與該子節點交換
    \item 繼續向下檢查,直到滿足條件或到達葉節點
\end{enumerate}
\end{block}

\vspace{1em}
\textbf{最壞情況:}元素需要移動到葉節點,需要 $O(\log n)$ 時間

\vspace{1em}
\textbf{待補充:刪除操作圖解與程式碼實作}
\end{frame}

\begin{frame}{建堆操作 (Build Heap)}
\begin{itemize}
    \item 將任意陣列轉換成堆積
    \item 從最後一個非葉節點開始,向前進行下沉操作
    \item 最後一個非葉節點的索引:$\lfloor n/2 \rfloor$
\end{itemize}

\vspace{1em}
\begin{block}{建堆演算法}
\begin{enumerate}
    \item 從索引 $\lfloor n/2 \rfloor$ 開始
    \item 對每個節點執行下沉操作
    \item 向前移動到索引 1(根節點)
\end{enumerate}
\end{block}

\vspace{1em}
\textbf{時間複雜度分析:}
\begin{itemize}
    \item 看似 $O(n \log n)$,實際上是 $O(n)$
    \item 大部分節點都在樹的底層,下沉距離很短
\end{itemize}

\vspace{1em}
\textbf{待補充:建堆過程演示與複雜度分析}
\end{frame}

\section{優先佇列}

\begin{frame}{什麼是優先佇列?}
\begin{itemize}
    \item 不是先進先出(FIFO),而是按優先級服務
    \item 每個元素都有一個優先級
    \item 優先級高的元素先被處理
    \item 堆積是實作優先佇列的最佳資料結構
\end{itemize}

\vspace{1em}
\begin{block}{基本操作}
\begin{itemize}
    \item \textbf{Insert:}插入帶有優先級的元素
    \item \textbf{Extract-Max/Min:}取出優先級最高/最低的元素
    \item \textbf{Peek:}查看最高優先級元素
    \item \textbf{Change Priority:}修改元素的優先級
\end{itemize}
\end{block}

\vspace{1em}
\begin{center}
\renewcommand{\arraystretch}{1.3}
\begin{tabular}{|>{\centering\arraybackslash}m{3cm}|>{\centering\arraybackslash}m{2cm}|>{\centering\arraybackslash}m{2cm}|}
\hline
\textbf{實作方式} & \textbf{插入} & \textbf{取出最值} \\
\hline
未排序陣列 & $O(1)$ & $O(n)$ \\
\hline
已排序陣列 & $O(n)$ & $O(1)$ \\
\hline
堆積 & $O(\log n)$ & $O(\log n)$ \\
\hline
\end{tabular}
\end{center}
\end{frame}

\begin{frame}{優先佇列的應用}
\begin{itemize}
    \item \textbf{作業調度:}作業系統根據優先級調度程序
    \item \textbf{Dijkstra 演算法:}尋找最短路徑
    \item \textbf{A* 搜尋:}啟發式搜尋演算法
    \item \textbf{霍夫曼編碼:}資料壓縮演算法
    \item \textbf{事件模擬:}按時間順序處理事件
\end{itemize}

\vspace{1em}
\begin{block}{實際例子}
\begin{itemize}
    \item 醫院急診室的病患排隊
    \item 印表機工作佇列
    \item 網路封包路由
\end{itemize}
\end{block}

\vspace{1em}
\textbf{待補充:優先佇列應用實例與程式碼}
\end{frame}

\section{堆積排序}

\begin{frame}{堆積排序 (Heap Sort)}
\begin{itemize}
    \item 利用堆積資料結構進行排序
    \item 分為兩個階段:建堆和排序
    \item 原地排序演算法,空間複雜度 $O(1)$
    \item 時間複雜度穩定為 $O(n \log n)$
\end{itemize}

\vspace{1em}
\begin{block}{堆積排序演算法}
\begin{enumerate}
    \item \textbf{建堆階段:}將陣列轉換成最大堆積
    \item \textbf{排序階段:}
    \begin{itemize}
        \item 將根節點(最大值)與最後一個元素交換
        \item 減少堆積大小
        \item 對根節點進行下沉操作
        \item 重複直到堆積大小為 1
    \end{itemize}
\end{enumerate}
\end{block}

\vspace{1em}
\textbf{結果:}遞增排序的陣列
\end{frame}

\begin{frame}{堆積排序的特性}
\begin{itemize}
    \item \textbf{時間複雜度:}$O(n \log n)$(所有情況)
    \item \textbf{空間複雜度:}$O(1)$(原地排序)
    \item \textbf{穩定性:}不穩定
    \item \textbf{適應性:}不是自適應演算法
\end{itemize}

\vspace{1em}
\begin{block}{與其他排序比較}
\begin{itemize}
    \item vs 快速排序:時間複雜度更穩定,但平均情況較慢
    \item vs 合併排序:空間複雜度更好,但不穩定
    \item vs 插入排序:大資料效率更高,但小資料較慢
\end{itemize}
\end{block}

\vspace{1em}
\textbf{適用場景:}
\begin{itemize}
    \item 需要穩定的 $O(n \log n)$ 效能
    \item 記憶體使用受限
    \item 不要求穩定性
\end{itemize}

\vspace{1em}
\textbf{待補充:堆積排序完整演示與程式碼}
\end{frame}

\section{堆積的變形與應用}

\begin{frame}{二項堆積 (Binomial Heap)}
\begin{itemize}
    \item 由多個二項樹組成的堆積結構
    \item 支援快速的合併操作
    \item 每個二項樹滿足堆積性質
\end{itemize}

\vspace{1em}
\begin{block}{二項樹性質}
\begin{itemize}
    \item $B_0$ 只有一個節點
    \item $B_k$ 由兩個 $B_{k-1}$ 組成
    \item $B_k$ 有 $2^k$ 個節點,高度為 $k$
\end{itemize}
\end{block}

\vspace{1em}
\textbf{操作複雜度:}
\begin{itemize}
    \item 合併:$O(\log n)$
    \item 插入:$O(\log n)$
    \item 取最小值:$O(\log n)$
\end{itemize}

\vspace{1em}
\textbf{待補充:二項堆積結構圖與合併操作}
\end{frame}

\begin{frame}{費波那契堆積 (Fibonacci Heap)}
\begin{itemize}
    \item 更進階的堆積結構
    \item 支援分攤 $O(1)$ 的插入和減少鍵值操作
    \item 在圖演算法中有重要應用
\end{itemize}

\vspace{1em}
\begin{block}{主要優勢}
\begin{itemize}
    \item 插入:$O(1)$ 分攤時間
    \item 合併:$O(1)$ 時間
    \item 減少鍵值:$O(1)$ 分攤時間
    \item 取最小值:$O(\log n)$ 分攤時間
\end{itemize}
\end{block}

\vspace{1em}
\textbf{應用:}
\begin{itemize}
    \item Dijkstra 最短路徑演算法
    \item Prim 最小生成樹演算法
\end{itemize}

\vspace{1em}
\textbf{待補充:費波那契堆積在圖演算法中的應用}
\end{frame}

\begin{frame}{堆積的實際應用舉例}
\begin{itemize}
    \item \textbf{Top-K 問題:}找出最大或最小的 K 個元素
    \item \textbf{即時系統:}任務調度和截止時間管理
    \item \textbf{資料流處理:}維護滑動視窗的統計資訊
    \item \textbf{圖演算法:}最短路徑、最小生成樹
\end{itemize}

\vspace{1em}
\begin{block}{Top-K 問題解法}
\begin{itemize}
    \item 使用大小為 K 的最小堆積
    \item 遍歷所有元素,維護堆積中的最大 K 個元素
    \item 時間複雜度:$O(n \log k)$
    \item 空間複雜度:$O(k)$
\end{itemize}
\end{block}

\vspace{1em}
\textbf{待補充:Top-K 問題詳細解法與程式碼}
\end{frame}

\begin{frame}{堆積實作的注意事項}
\begin{itemize}
    \item \textbf{索引選擇:}從 0 或 1 開始各有優缺點
    \item \textbf{比較函數:}需要一致性,支援自定義比較
    \item \textbf{容量管理:}動態調整陣列大小
    \item \textbf{邊界檢查:}避免陣列越界
\end{itemize}

\vspace{1em}
\begin{block}{常見錯誤}
\begin{itemize}
    \item 索引計算錯誤
    \item 堆積性質維護不當
    \item 比較函數不一致
    \item 忘記更新堆積大小
\end{itemize}
\end{block}

\vspace{1em}
\textbf{最佳實踐:}
\begin{itemize}
    \item 使用現成的優先佇列庫(如 C++ \texttt{priority\_queue})
    \item 充分測試邊界情況
    \item 考慮使用泛型實作
\end{itemize}
\end{frame}

\begin{frame}{本章總結}
\begin{itemize}
    \item \textbf{堆積:}完全二元樹,滿足堆積性質
    \item \textbf{基本操作:}插入、刪除、建堆,時間複雜度 $O(\log n)$ 或更好
    \item \textbf{優先佇列:}堆積是最佳實作方式
    \item \textbf{堆積排序:}穩定 $O(n \log n)$,原地排序
    \item \textbf{進階堆積:}二項堆積、費波那契堆積
    \item \textbf{應用:}Top-K、調度、圖演算法、事件模擬
\end{itemize}

\vspace{1em}
\begin{center}
    \textbf{堆積是優先級處理的最佳工具!}
\end{center}
\end{frame}

\end{document}