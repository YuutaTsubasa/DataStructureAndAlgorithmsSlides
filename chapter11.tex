\documentclass{beamer}
\usetheme{Madrid}

\usepackage{xeCJK}
\usepackage{fontspec}
\setmainfont[
  Path = fonts/,
  UprightFont = *-Regular,
  BoldFont = *-Bold,
  ItalicFont = *-Medium,
  BoldItalicFont = *-Bold
]{NotoSansTC}
\setCJKmainfont[
  Path = fonts/,
  UprightFont = *-Regular,
  BoldFont = *-Bold,
  ItalicFont = *-Medium,
  BoldItalicFont = *-Bold
]{NotoSansTC}

\usepackage{amsmath}
\usepackage{graphicx}
\usepackage{listings}
\usepackage{xcolor}
\usepackage{array}

\lstdefinestyle{cppstyle}{
  language=C++,
  basicstyle=\ttfamily\small,
  keywordstyle=\color{blue}\bfseries,
  commentstyle=\color{gray},
  stringstyle=\color{orange},
  numbers=left,
  numberstyle=\tiny\color{gray},
  stepnumber=1,
  numbersep=5pt,
  frame=single,
  breaklines=true,
  showstringspaces=false,
  tabsize=2
}

\begin{document}

\title{資料結構與演算法入門:第 11 章}
\subtitle{雜項 (Miscellaneous)}
\author{悠太翼 Yuuta Tsubasa}
\date{\today}

\frame{\titlepage}

\begin{frame}{本章內容概覽}
\begin{itemize}
    \item 本章介紹一些重要但較為獨立的主題
    \item 涵蓋實際程式開發中常遇到的問題
    \item 補充前面章節未涵蓋的重要概念
\end{itemize}

\vspace{1em}
\begin{block}{主要內容}
\begin{itemize}
    \item \textbf{大數運算 (Big Number):}處理超出基本資料型別範圍的數值
    \item \textbf{雜湊 (Hash):}快速查找和資料組織的重要技術
\end{itemize}
\end{block}

\vspace{1em}
\begin{center}
\textbf{這些技術在現代程式開發中無處不在!}
\end{center}
\end{frame}

\section{大數運算}

\begin{frame}{為什麼需要大數運算?}
\begin{itemize}
    \item 標準整數類型有範圍限制
    \begin{itemize}
        \item 32 位元 int:約 $\pm 2 \times 10^9$
        \item 64 位元 long long:約 $\pm 9 \times 10^{18}$
    \end{itemize}
    \item 許多實際問題需要更大的數值
    \item 加密演算法、科學計算、競程常遇到
\end{itemize}

\vspace{1em}
\begin{block}{大數應用場景}
\begin{itemize}
    \item 計算大階乘:$100!$、$1000!$
    \item RSA 加密:處理數百位元的質數
    \item 費波那契數列的大項
    \item 組合數學中的大數計算
    \item 精確的浮點數運算
\end{itemize}
\end{block}

\vspace{1em}
\textbf{範例:}$100! \approx 9.33 \times 10^{157}$,遠超 64 位元整數範圍
\end{frame}

\begin{frame}{大數的表示方法}
\begin{itemize}
    \item \textbf{字串表示:}直觀但運算不便
    \item \textbf{陣列表示:}每個元素存放一位或多位數字
    \item \textbf{向量表示:}動態調整大小,較為常用
\end{itemize}

\vspace{1em}
\begin{block}{陣列表示的考量}
\begin{itemize}
    \item \textbf{進位制選擇:}十進位 vs 更高進位制
    \item \textbf{位數順序:}低位在前 vs 高位在前
    \item \textbf{符號處理:}正負數的表示方式
    \item \textbf{前導零:}避免不必要的存儲
\end{itemize}
\end{block}

\vspace{1em}
\textbf{常見表示:}
\begin{itemize}
    \item 每個陣列元素存放一位十進位數字
    \item 低位存放在陣列前端(便於進位處理)
    \item 使用額外變數記錄符號
\end{itemize}

\vspace{1em}
\textbf{待補充:大數表示法的程式碼實作}
\end{frame}

\begin{frame}{大數基本運算:加法}
\begin{itemize}
    \item 模擬手工加法的過程
    \item 從低位開始,逐位相加並處理進位
    \item 需要考慮不同長度的數字
\end{itemize}

\vspace{1em}
\begin{block}{加法演算法步驟}
\begin{enumerate}
    \item 從最低位開始
    \item 將對應位數字和進位相加
    \item 如果和 ≥ 10,產生新的進位
    \item 將個位數存入結果
    \item 處理下一位,直到所有位數完成
    \item 處理最後的進位
\end{enumerate}
\end{block}

\vspace{1em}
\textbf{時間複雜度:}$O(\max(m, n))$,$m$、$n$ 為兩數的位數\\
\textbf{空間複雜度:}$O(\max(m, n))$

\vspace{1em}
\textbf{待補充:大數加法的詳細實作與範例}
\end{frame}

\begin{frame}{大數基本運算:乘法}
\begin{itemize}
    \item 最直接的方法是模擬手工乘法
    \item 每位乘以每位,累加到正確位置
    \item 比加法複雜得多
\end{itemize}

\vspace{1em}
\begin{block}{簡單乘法演算法}
\begin{enumerate}
    \item 初始化結果為 0
    \item 對於第二個數的每一位 $d_i$:
    \begin{itemize}
        \item 將第一個數乘以 $d_i$
        \item 將結果左移 $i$ 位
        \item 加到總結果中
    \end{itemize}
    \item 處理所有進位
\end{enumerate}
\end{block}

\vspace{1em}
\textbf{時間複雜度:}$O(mn)$(簡單演算法)\\
\textbf{進階演算法:}Karatsuba 乘法 $O(n^{\log_2 3}) \approx O(n^{1.58})$

\vspace{1em}
\textbf{待補充:大數乘法實作與 Karatsuba 演算法介紹}
\end{frame}

\begin{frame}{大數的其他運算}
\begin{itemize}
    \item \textbf{減法:}類似加法,但需要處理借位
    \item \textbf{除法:}最複雜,通常使用長除法
    \item \textbf{比較:}從高位開始比較
    \item \textbf{次方:}使用快速冪演算法
\end{itemize}

\vspace{1em}
\begin{block}{大數除法要點}
\begin{itemize}
    \item 試商:估算每一位的商
    \item 調整:如果估算過大,需要調整
    \item 效率考量:可能需要二分搜尋優化
\end{itemize}
\end{block}

\vspace{1em}
\textbf{快速冪:}$a^n$ 可在 $O(\log n)$ 次乘法內完成\\
\textbf{應用:}RSA 加密中的模冪運算

\vspace{1em}
\textbf{待補充:大數其他運算的實作方法}
\end{frame}

\begin{frame}{大數函式庫}
\begin{itemize}
    \item 實作大數運算複雜且容易出錯
    \item 建議使用現成的函式庫
    \item 不同語言有不同的選擇
\end{itemize}

\vspace{1em}
\begin{center}
\renewcommand{\arraystretch}{1.3}
\begin{tabular}{|>{\centering\arraybackslash}m{2.5cm}|>{\raggedright\arraybackslash}m{6cm}|}
\hline
\textbf{程式語言} & \textbf{大數函式庫} \\
\hline
C++ & GMP, Boost.Multiprecision \\
\hline
Java & BigInteger, BigDecimal \\
\hline
Python & 內建支援(int 自動轉換) \\
\hline
C\# & BigInteger (.NET) \\
\hline
JavaScript & BigInt (ES2020) \\
\hline
\end{tabular}
\end{center}

\vspace{1em}
\textbf{注意事項:}
\begin{itemize}
    \item 效能考量:大數運算比基本類型慢很多
    \item 記憶體使用:大數佔用更多空間
    \item 競程限制:部分競賽不允許使用外部函式庫
\end{itemize}
\end{frame}

\section{雜湊 (Hash)}

\begin{frame}{什麼是雜湊?}
\begin{itemize}
    \item 雜湊(Hash)是將任意大小的資料映射到固定大小的技術
    \item 雜湊函數:$h: U \rightarrow \{0, 1, 2, ..., m-1\}$
    \item 目標:將鍵值均勻分布到雜湊表中
    \item 核心優勢:平均 $O(1)$ 的查找時間
\end{itemize}

\vspace{1em}
\begin{block}{雜湊的應用}
\begin{itemize}
    \item \textbf{雜湊表:}快速的鍵值查找
    \item \textbf{資料完整性:}檢測資料是否被修改
    \item \textbf{密碼學:}安全雜湊函數
    \item \textbf{去重複:}快速判斷元素是否重複
    \item \textbf{快取:}Web 快取、資料庫快取
\end{itemize}
\end{block}

\vspace{1em}
\textbf{理想特性:}均勻分布、計算快速、雜湊值固定長度
\end{frame}

\begin{frame}{雜湊函數設計}
\begin{itemize}
    \item 好的雜湊函數應該將鍵值均勻分布
    \item 避免過多的碰撞(collision)
    \item 計算效率要高
\end{itemize}

\vspace{1em}
\begin{block}{常見雜湊函數}
\begin{itemize}
    \item \textbf{除法雜湊:}$h(k) = k \bmod m$
    \item \textbf{乘法雜湊:}$h(k) = \lfloor m \cdot (k \cdot A \bmod 1) \rfloor$
    \item \textbf{字串雜湊:}多項式雜湊
    \item \textbf{通用雜湊:}隨機選擇參數的函數族
\end{itemize}
\end{block}

\vspace{1em}
\textbf{字串雜湊範例:}
$$h(s) = \left( \sum_{i=0}^{n-1} s[i] \cdot p^i \right) \bmod m$$
其中 $p$ 是質數(通常選 31 或 37),$m$ 是大質數

\vspace{1em}
\textbf{待補充:各種雜湊函數的實作與特性分析}
\end{frame}

\begin{frame}{雜湊碰撞處理}
\begin{itemize}
    \item 碰撞:不同的鍵值產生相同的雜湊值
    \item 由鴿籠原理,碰撞不可避免
    \item 需要有效的碰撞處理策略
\end{itemize}

\vspace{1em}
\begin{block}{鏈結法(Chaining)}
\begin{itemize}
    \item 每個雜湊位置維護一個連結串列
    \item 碰撞的元素加入同一個串列
    \item 查找時需要遍歷串列
\end{itemize}
\end{block}

\vspace{1em}
\begin{block}{開放定址法(Open Addressing)}
\begin{itemize}
    \item 碰撞時尋找下一個空位
    \item \textbf{線性探測:}依序檢查下一個位置
    \item \textbf{二次探測:}以二次函數跳躍
    \item \textbf{雙重雜湊:}使用第二個雜湊函數
\end{itemize}
\end{block}

\vspace{1em}
\textbf{待補充:碰撞處理方法的圖解與性能比較}
\end{frame}

\begin{frame}{雜湊表性能分析}
\begin{itemize}
    \item 負載因子:$\alpha = \frac{n}{m}$(元素數/表格大小)
    \item 性能與負載因子密切相關
    \item 需要動態調整表格大小
\end{itemize}

\vspace{1em}
\begin{center}
\renewcommand{\arraystretch}{1.3}
\begin{tabular}{|>{\centering\arraybackslash}m{3cm}|>{\centering\arraybackslash}m{2cm}|>{\centering\arraybackslash}m{2cm}|}
\hline
\textbf{碰撞處理方法} & \textbf{平均查找} & \textbf{最壞查找} \\
\hline
鏈結法 & $O(1 + \alpha)$ & $O(n)$ \\
\hline
線性探測 & $O\left(\frac{1}{1-\alpha}\right)$ & $O(n)$ \\
\hline
\end{tabular}
\end{center}

\vspace{1em}
\begin{block}{表格調整}
\begin{itemize}
    \item 當負載因子過高時,擴大表格
    \item 重新雜湊所有元素(rehashing)
    \item 分攤分析:擴大操作的平均成本仍為 $O(1)$
\end{itemize}
\end{block}

\vspace{1em}
\textbf{待補充:負載因子對性能的影響與動態調整策略}
\end{frame}

\begin{frame}{雜湊在字串處理中的應用}
\begin{itemize}
    \item 字串雜湊是處理字串問題的重要技巧
    \item 可以快速比較字串、尋找模式
    \item Rolling Hash 技術特別有用
\end{itemize}

\vspace{1em}
\begin{block}{Rolling Hash}
\begin{itemize}
    \item 在滑動視窗中快速更新雜湊值
    \item $O(1)$ 時間計算新的雜湊值
    \item 應用:字串匹配、回文檢測
\end{itemize}
\end{block}

\vspace{1em}
\textbf{範例:}在文本中尋找模式字串
\begin{enumerate}
    \item 計算模式字串的雜湊值
    \item 滑動視窗遍歷文本,計算每個子字串的雜湊值
    \item 雜湊值相同時,進行實際字串比較
\end{enumerate}

\vspace{1em}
\textbf{時間複雜度:}平均 $O(n + m)$,最壞 $O(nm)$

\vspace{1em}
\textbf{待補充:Rolling Hash 的數學原理與實作}
\end{frame}

\begin{frame}{密碼學雜湊函數}
\begin{itemize}
    \item 與一般雜湊函數有不同的安全性要求
    \item 用於密碼存儲、數位簽章、區塊鏈等
    \item 需要滿足更嚴格的性質
\end{itemize}

\vspace{1em}
\begin{block}{安全性要求}
\begin{itemize}
    \item \textbf{單向性:}難以從雜湊值推導原始資料
    \item \textbf{抗碰撞性:}難以找到產生相同雜湊值的不同輸入
    \item \textbf{雪崩效應:}輸入的小改變導致雜湊值大幅變化
\end{itemize}
\end{block}

\vspace{1em}
\begin{block}{常見密碼學雜湊函數}
\begin{itemize}
    \item \textbf{MD5:}已被破解,不建議使用
    \item \textbf{SHA-1:}安全性存疑,逐漸淘汰
    \item \textbf{SHA-256:}目前廣泛使用
    \item \textbf{SHA-3:}最新標準
\end{itemize}
\end{block}

\vspace{1em}
\textbf{待補充:密碼學雜湊函數的應用實例}
\end{frame}

\begin{frame}{雜湊表的實際應用}
\begin{itemize}
    \item 幾乎所有程式語言都內建雜湊表
    \item 是現代程式設計的基礎資料結構
    \item 了解原理有助於最佳化性能
\end{itemize}

\vspace{1em}
\begin{center}
\renewcommand{\arraystretch}{1.3}
\begin{tabular}{|>{\centering\arraybackslash}m{2.5cm}|>{\raggedright\arraybackslash}m{6cm}|}
\hline
\textbf{程式語言} & \textbf{雜湊表實作} \\
\hline
C++ & std::unordered\_map, std::unordered\_set \\
\hline
Java & HashMap, HashSet \\
\hline
Python & dict, set \\
\hline
JavaScript & Object, Map, Set \\
\hline
C\# & Dictionary, HashSet \\
\hline
\end{tabular}
\end{center}

\vspace{1em}
\textbf{使用建議:}
\begin{itemize}
    \item 優先使用語言內建的雜湊表
    \item 注意鍵的雜湊函數品質
    \item 考慮負載因子對性能的影響
    \item 在性能關鍵的場景中自定義雜湊函數
\end{itemize}
\end{frame}

\begin{frame}{本章總結}
\begin{itemize}
    \item \textbf{大數運算:}
    \begin{itemize}
        \item 處理超出基本類型範圍的數值計算
        \item 基本運算:加、減、乘、除、比較
        \item 實用建議:使用成熟的函式庫
    \end{itemize}
    \item \textbf{雜湊技術:}
    \begin{itemize}
        \item 快速查找的核心技術
        \item 雜湊函數設計與碰撞處理
        \item 廣泛應用:資料結構、字串處理、密碼學
    \end{itemize}
    \item 這些技術在實際開發中無處不在
    \item 理解原理有助於寫出更高效的程式
\end{itemize}

\vspace{1em}
\begin{center}
    \textbf{掌握這些實用技術,讓程式設計更加得心應手!}
\end{center}
\end{frame}

\begin{frame}{課程總結}
\begin{itemize}
    \item 我們已經完成了資料結構與演算法的基礎學習
    \item 從複雜度分析到圖論,涵蓋了核心主題
    \item 這些知識是進一步學習的重要基礎
\end{itemize}

\vspace{1em}
\begin{block}{學習建議}
\begin{itemize}
    \item \textbf{多練習:}理論要結合實作
    \item \textbf{解決實際問題:}參與競程、開源專案
    \item \textbf{持續學習:}演算法是不斷發展的領域
    \item \textbf{思考應用:}將所學應用到實際專案中
\end{itemize}
\end{block}

\vspace{1em}
\begin{center}
    \textbf{恭喜完成課程!}\\
    \textbf{繼續在演算法的世界中探索吧!}
\end{center}
\end{frame}

\end{document}