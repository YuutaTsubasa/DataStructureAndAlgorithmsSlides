\documentclass{beamer}
\usetheme{Madrid}

\usepackage[utf8]{inputenc}
\usepackage{xeCJK}
\setCJKmainfont{Noto Sans CJK TC}

\usepackage{listings}
\usepackage{xcolor}

\lstset{
  literate={<<}{{\textless{}\textless{}}}2 {>>}{{\textgreater{}\textgreater{}}}2
}

\lstdefinestyle{cppstyle}{
  language=C++,
  basicstyle=\ttfamily\small,
  keywordstyle=\color{blue}\bfseries,
  commentstyle=\color{gray},
  stringstyle=\color{orange},
  numbers=left,
  numberstyle=\tiny\color{gray},
  stepnumber=1,
  numbersep=5pt,
  frame=single,
  breaklines=true,
  showstringspaces=false,
  tabsize=2
}

\begin{document}

\title{資料結構與演算法入門:第 2 章}
\subtitle{窮舉法與遞迴}
\author{悠太翼 Yuuta Tsubasa}
\date{\today}

\frame{\titlepage}

\begin{frame}{什麼是窮舉法?}
\begin{itemize}
    \item 窮舉法(Brute Force):把所有可能的情況都試一遍
    \item 是最直覺、最通用的解法
    \item 缺點:效率可能非常低,無法處理太大的輸入空間
    \item 優點:實作簡單、一定能找出正確答案(若有)
\end{itemize}
\end{frame}

\begin{frame}[fragile]{範例一:索尼克比賽示範關卡安排}
\begin{itemize}
    \item 關卡列表:[Green Hill, Seaside Hill, City Escape]
    \item 玩家列表:[SonicBoss, Jerry, Zexas]
    \item 從中挑一個人來玩一關,列出所有可能的組合:
\end{itemize}

\begin{lstlisting}[style=cppstyle]
vector<string> stages = {"Green Hill", "Seaside Hill", "City Escape"};
vector<string> players = {"SonicBoss", "Jerry", "Zexas"};

for (string stage : stages) {
    for (string player : players) {
        cout << stage << " - " << player << endl;
    }
}
\end{lstlisting}

\begin{itemize}
    \item 共列出 $3 \times 3 = 9$ 種組合
    \item 時間複雜度:$O(n \times m)$,其中 $n$ 是關卡數,$m$ 是玩家數
\end{itemize}
\end{frame}

\end{document}