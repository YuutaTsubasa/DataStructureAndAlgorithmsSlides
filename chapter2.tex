\documentclass{beamer}
\usetheme{Madrid}

\usepackage[utf8]{inputenc}
\usepackage{xeCJK}
\usepackage{fontspec}
\setmainfont[
  Path = fonts/,
  UprightFont = *-Regular,
  BoldFont = *-Bold,
  ItalicFont = *-Medium,
  BoldItalicFont = *-Bold
]{NotoSansTC}
\setCJKmainfont[
  Path = fonts/,
  UprightFont = *-Regular,
  BoldFont = *-Bold,
  ItalicFont = *-Medium,
  BoldItalicFont = *-Bold
]{NotoSansTC}

\usepackage{listings}
\usepackage{xcolor}
\usepackage{array}
\usepackage{tikz}
\usepackage{forest}

\lstset{
  literate={<<}{{\textless{}\textless{}}}2 {>>}{{\textgreater{}\textgreater{}}}2
}

\lstdefinestyle{cppstyle}{
  language=C++,
  basicstyle=\ttfamily\small,
  keywordstyle=\color{blue}\bfseries,
  commentstyle=\color{gray},
  stringstyle=\color{orange},
  numbers=left,
  numberstyle=\tiny\color{gray},
  stepnumber=1,
  numbersep=5pt,
  frame=single,
  breaklines=true,
  showstringspaces=false,
  tabsize=2
}

\begin{document}

\title{資料結構與演算法入門:第 2 章}
\subtitle{窮舉法與遞迴}
\author{悠太翼 Yuuta Tsubasa}
\date{\today}

\frame{\titlepage}

\begin{frame}{什麼是窮舉法?}
\begin{itemize}
    \item 窮舉法(Brute Force):把所有可能的情況都試一遍
    \item 是最直覺、最通用的解法
    \item 適合問題空間不大的場合
\end{itemize}

\vspace{1em}

\begin{center}
\renewcommand{\arraystretch}{1.4}
\begin{tabular}{|>{\centering\arraybackslash}m{3cm}|>{\raggedright\arraybackslash}m{6.5cm}|}
\hline
\textbf{優點} & 
實作簡單 \newline
一定能找出正確答案(若有) \newline
適用於所有問題類型 \\
\hline
\textbf{缺點} & 
效率低,無法處理太大輸入 \newline
計算量可能爆炸性成長 \newline
缺乏最佳化 \\
\hline
\end{tabular}
\end{center}
\end{frame}

\begin{frame}[fragile]{範例一:索尼克比賽示範關卡安排}
\begin{itemize}
    \item 關卡列表:[Green Hill, Seaside Hill, City Escape]
    \item 玩家列表:[SonicBoss, Jerry, Zexas]
    \item 從中挑一個人來玩一關,列出所有可能的組合:
\end{itemize}

\begin{lstlisting}[style=cppstyle]
vector<string> stages = {"Green Hill", "Seaside Hill", "City Escape"};
vector<string> players = {"SonicBoss", "Jerry", "Zexas"};

for (string stage : stages) {
    for (string player : players) {
        cout << stage << " - " << player << endl;
    }
}
\end{lstlisting}

\begin{itemize}
    \item 共列出 $3 \times 3 = 9$ 種組合
    \item 時間複雜度:$O(n \times m)$,其中 $n$ 是關卡數,$m$ 是玩家數
\end{itemize}
\end{frame}

\begin{frame}[fragile]{範例二:找出陣列中兩個數的和為指定值}
\begin{itemize}
    \item 題目:給定一個整數陣列和一個目標值,找出陣列中是否存在兩個數字相加等於目標值。
    \item 這是一個典型的窮舉法適用問題(暴力解法):
\end{itemize}

\vspace{0.5em}
\begin{lstlisting}[style=cppstyle]
vector<int> nums = {1, 3, 5, 7, 9};
int target = 10;

bool isFound = false;
for (int i = 0; i < nums.size(); i++) {
    for (int j = i + 1; j < nums.size(); j++) {
        if (nums[i] + nums[j] == target) {
            isFound = true;
        }
    }
}
\end{lstlisting}

\vspace{0.5em}
\begin{itemize}
    \item 時間複雜度:$O(n^2)$,需要兩層迴圈比較所有配對
    \item 空間複雜度:$O(1)$,僅使用常數額外變數
\end{itemize}
\end{frame}

\begin{frame}[fragile]{加速方法:使用 target - num 的概念}
\begin{itemize}
    \item 窮舉法效率低,能不能更快?
    \item 我們可以「邊走邊記錄」,用一個集合記住看過的數字
    \item 每次檢查是否存在目標值減去目前數字的差值
\end{itemize}

\vspace{0.5em}
\begin{lstlisting}[style=cppstyle]
vector<int> nums = {1, 3, 5, 7, 9};
int target = 10;
unordered_set<int> seen;

bool isFound = false;
for (int num : nums) {
    if (seen.find(target - num) != seen.end()) {
        // 找到了 num + (target - num) = target
        isFound = true;
    }
    seen.insert(num);
}
\end{lstlisting}

\vspace{0.5em}
\begin{itemize}
    \item 時間複雜度:$O(n)$,每個數字最多操作一次
    \item 空間複雜度:$O(n)$,額外使用一個集合儲存已看過的數字
\end{itemize}
\end{frame}

\begin{frame}[fragile]{什麼是遞迴?}
\begin{itemize}
    \item 遞迴(Recursion)是一種函式「呼叫自己」的寫法
    \item 一個遞迴函式必須要有:
    \begin{itemize}
        \item \textbf{基底情況(Base Case):} 不再呼叫自己,直接回傳
        \item \textbf{遞迴情況(Recursive Case):} 將問題縮小,呼叫自己來解更小的子問題
    \end{itemize}
    \item 遞迴常用於:
    \begin{itemize}
        \item 重複的結構(如樹、圖、排列組合)
        \item 難以用迴圈表達的邏輯
    \end{itemize}
\end{itemize}

\vspace{0.5em}
\begin{block}{舉例:從 1 數到 n}
\begin{lstlisting}[style=cppstyle]
void count(int n) {
    if (n == 0) return;
    count(n - 1);
    cout << n << endl;
}
\end{lstlisting}
\end{block}
\end{frame}

\begin{frame}{遞迴流程圖:count(3)}
\centering
\begin{tikzpicture}[
    node distance=1.6cm,
    boxnode/.style={draw, rounded corners, minimum width=3cm, minimum height=1cm, font=\small},
    textnode/.style={font=\tiny},
    arrow/.style={->, thick}
]

\node[boxnode] (c3) {count(3)};
\node[boxnode] (c2) [below of=c3] {count(2)};
\node[boxnode] (c1) [below of=c2] {count(1)};
\node[boxnode] (c0) [below of=c1] {count(0)};

% 呼叫箭頭(向下彎)
\draw[arrow, bend right=15] (c3) to node[textnode, pos=0.5, xshift=-10pt] {呼叫} (c2);
\draw[arrow, bend right=15] (c2) to node[textnode, pos=0.5, xshift=-10pt] {呼叫} (c1);
\draw[arrow, bend right=15] (c1) to node[textnode, pos=0.5, xshift=-10pt] {呼叫} (c0);

% 返回箭頭(向上彎)
\draw[arrow, bend right=15] (c0) to node[textnode, pos=0.5, xshift=20pt] {返回,印出 1} (c1);
\draw[arrow, bend right=15] (c1) to node[textnode, pos=0.5, xshift=20pt] {返回,印出 2} (c2);
\draw[arrow, bend right=15] (c2) to node[textnode, pos=0.5, xshift=20pt] {返回,印出 3} (c3);

\end{tikzpicture}

\vspace{1em}
\textbf{說明:} 遞迴先呼叫到底,再從 \texttt{count(0)} 返回並印出數字
\end{frame}

\begin{frame}[fragile]{遞迴寫法(上):列出每個關卡}
\begin{itemize}
    \item 第一層遞迴:逐一處理每個關卡
    \item 先不處理玩家部分,只列出關卡
\end{itemize}

\vspace{0.5em}
\begin{lstlisting}[style=cppstyle]
vector<string> stages = {"Green Hill", "Seaside Hill", "City Escape"};

void listStages(int si) {
    if (si == stages.size()) return;
    cout << stages[si] << endl;
    listStages(si + 1);
}

listStages(0);
\end{lstlisting}

\vspace{0.5em}
\begin{itemize}
    \item 遞迴終止條件為索引超出範圍
    \item 時間複雜度:$O(n)$,其中 $n$ 是關卡數
\end{itemize}
\end{frame}

\begin{frame}[fragile]{遞迴寫法(下):列出所有關卡與玩家的組合}
\begin{itemize}
    \item 第二層遞迴:處理每個關卡下的所有玩家
\end{itemize}

\vspace{0.5em}
\begin{lstlisting}[style=cppstyle]
vector<string> players = {"SonicBoss", "Jerry", "Zexas"};

void listPlayers(int si, int pi) {
    if (pi == players.size()) return;
    cout << stages[si] << " - " << players[pi] << endl;
    listPlayers(si, pi + 1);
}

void listCombinations(int si) {
    if (si == stages.size()) return;
    listPlayers(si, 0);
    listCombinations(si + 1);
}

listCombinations(0);
\end{lstlisting}

\vspace{0.5em}
\begin{itemize}
    \item 雙層遞迴對應雙層迴圈,總時間複雜度為 $O(n \times m)$
\end{itemize}
\end{frame}

\begin{frame}[fragile]{遞迴寫法:找出兩數和為目標值(窮舉)}
\begin{itemize}
    \item 使用兩層遞迴枚舉所有組合
\end{itemize}

\vspace{0.5em}
\begin{lstlisting}[style=cppstyle]
bool findSumBrute(vector<int>& nums, int i, int j, int target) {
    if (i >= nums.size()) return false;
    if (j >= nums.size()) return findSumBrute(nums, i + 1, i + 2, target);
    if (nums[i] + nums[j] == target) return true;
    return findSumBrute(nums, i, j + 1, target);
}

bool isFound = findSumBrute(nums, 0, 1, 10);
\end{lstlisting}

\vspace{0.5em}
\begin{itemize}
    \item 每對 $(i, j)$ 都嘗試一次,時間複雜度為 $O(n^2)$
\end{itemize}
\end{frame}

\begin{frame}{什麼是 Backtracking?}
\begin{itemize}
    \item Backtracking(回溯法)是一種系統性嘗試所有可能解的技巧。
    \item 當某一步發現不可行,就會「回溯」到上一步,嘗試其他路線。
    \item 常用於加速去解決需要列舉所有可能配置的問題(如排列組合、棋盤問題)。
    \item 常搭配遞迴使用。
\end{itemize}

\vspace{1em}

\begin{center}
\renewcommand{\arraystretch}{1.3}
\begin{tabular}{|>{\centering\arraybackslash}m{3cm}|>{\raggedright\arraybackslash}m{6.5cm}|}
\hline
\textbf{優點} &
能有效減少不必要的搜尋路徑 \newline
適合解決組合、排列、棋盤等問題 \\
\hline
\textbf{缺點} &
最壞情況下仍需窮舉所有可能 \newline
實作較複雜,需要設計剪枝條件 \\
\hline
\end{tabular}
\end{center}
\end{frame}

\begin{frame}{範例一:解數獨}
\textbf{題目:} 給定一個 9×9 的數獨盤面,請填入數字使其符合以下規則:
\begin{itemize}
    \item 每列、每行、每個 3×3 區塊內不得有重複數字(1–9)
\end{itemize}

\vspace{0.5em}
\textbf{解法概念:}
\begin{itemize}
    \item 從左上角開始,對每個空格嘗試放 1–9
    \item 若不違反規則就遞迴嘗試下一格
    \item 若後續無法完成,就「回溯」並嘗試其他數字
\end{itemize}
\end{frame}

\begin{frame}[fragile]{數獨遞迴解法}
\begin{lstlisting}[style=cppstyle]
bool isValid(int row, int col, int num);
bool solveSudoku(int row, int col) {
    if (row == 9) return true; // 全部填完
    if (col == 9) return solveSudoku(row + 1, 0); // 換行

    if (board[row][col] != 0)
        return solveSudoku(row, col + 1);

    for (int num = 1; num <= 9; num++) {
        if (isValid(row, col, num)) {
            board[row][col] = num;
            if (solveSudoku(row, col + 1)) return true;
            board[row][col] = 0; // 回溯
        }
    }
    return false; // 這格無解
}
\end{lstlisting}
\end{frame}

\begin{frame}{數獨解法分析}
\begin{itemize}
    \item 數獨盤面為 $9 \times 9$,最多 81 個空格,每格最多嘗試 9 種數字
    \item \textbf{最壞情況時間複雜度:} $O(9^{81})$
    \item 實際上盤面大小固定,因此這個複雜度可以視為 $O(1)$(常數級)
    \item \textbf{實際運行通常更快,原因如下:}
    \begin{itemize}
        \item 限制條件強(列、行、九宮格不能重複) → 很多數字會被剪枝
        \item 可以加上技巧加速:
        \begin{itemize}
            \item 先預處理已知空格
            \item 空格依照可填數字數量排序(最少優先原則)
            \item 用 bitmask 等資料結構快速檢查合法性
        \end{itemize}
    \end{itemize}
    \item 延伸應用:若盤面為 $n \times n$ 數獨,則時間複雜度為 $O( (n^2)^n )$
\end{itemize}
\end{frame}


\begin{frame}[fragile]{後續待補......}
待續.....
\end{frame}
\end{document}